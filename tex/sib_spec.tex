\documentclass{ldbc}

\usepackage{hyperref}
\usepackage{multirow}
\usepackage{float}
\usepackage{amsfonts}
\usepackage{bbding}
\usepackage{xcolor}
\usepackage{rotating}
\usepackage{pifont} 
\usepackage{rotating}
\usepackage{subfigure}
\usepackage{amsmath}
\usepackage{lscape}
\usepackage{longtable}
\usepackage{listings}

\newcommand{\cmark}{\ding{51}}%
\newcommand{\xmark}{\ding{55}}%
\newcommand{\yes}{\color{green}\ding{51}\color{black}}
\newcommand{\no}{\color{red}\ding{55}\color{black}}

\newcommand{\cref}[1]{Chapter~\ref{#1}}
\newcommand{\sref}[1]{Section~\ref{#1}}
\newcommand{\tref}[1]{Table~\ref{#1}}
\newcommand{\fref}[1]{Figure~\ref{#1}}
\newcommand{\eref}[1]{Equation~\ref{#1}}
\newcommand{\aref}[1]{Appendix~\ref{#1}}

% Alex Averbuch: used internally only, to make missing/erroneous sections stand out
\newcommand{\alert}[1]{\textit{\textbf{{\color{red}#1}}}}

% todo change the following information as appropriate
%\WP{N/A}
\renewcommand{\wpIDText}{N/A}
\WPTitle{Social Network Benchmark Task Force}

%\delID{}
\renewcommand{\delIDText}{}
\delName{LDBC Social Network Benchmark (SNB) - v0.2 First Public Draft Release v0.2}

\dueDate{M12}
\submissionDate{M13}

%dissemination level
\dissPU % Public
%\dissRE % Restricted to group
%\dissPP % Restricted to programme
%\dissCO % Consortium-only

%nature
\natR % Report
%\natP % Prototype
%\natD % Demonstrator
%\natO % Other

\author{[Arnau Prat (UPC)]}
\authorPartner{Arnau Prat (UPC)}
\responsibleAuthor{Arnau Prat}
\responsiblePartner{UPC}
\responsiblePhone{+34934054032}
\responsibleEmail{aprat@ac.upc.edu}

% comment the following out if there are no contributors beside the main authors
\contributor{[Peter Boncz (VUA), Josep Llu\'is Larriba (UPC), Renzo
Angles (TALCA), Alex Averbuch (NEO), Orri Erling (OGL), Andrey
Gubichev (TUM), Mirko Spasi\'c (OGL)], Minh-Duc Pham (VUA), Norbert Mart\'inez (SPARSITY)}

\reviewerOne{\alert{???}}
\reviewerTwo{\alert{???}}

\keywords{benchmark, choke points, dataset generator, graph database, query set, RDF, workload, auditing rules, publication rules, scale factors}

% for version numbers, use 2 digits separated by a dot (First digit is
% 0 for ``draft'', 1 for ``project approved'', 2 for ``further revisions''
% such as when the EC rejected version 1
\versionLog{
    \versionLogEntry{19/05/2014}{0.1}{Arnau Prat}{First draft}
}
\lastVersion{0.1}

% uncomment the following for final version
%\final

\documentUrl{\url{https://svn.sti2.at/ldbc/trunk/sib_spec/sib_spec.tex/}}

\abstract{
LDBC's Social Network Benchmark (LDBC-SNB) is an effort intended to test
various functionalities of systems used for graph-like data management. For this,
LDBC-SNB uses the recognizable scenario of operating a social network, characterized by
its graph-shaped data.

LDBC-SNB consists of three sub-benchmarks, or workloads, that focus on different
functionalities. In this document, a preliminary version of the Interactive
Workload. The other
workloads, still in development, are the Business
Intelligence Workloads (with analytical queries), and the Graph Analytics
Workload (with graph algorithms).

This document contains a detailed explanation of the data used in the
whole LDBC-SNB benchmark, a detailed description for all the Interactive Workload
,and instructions on how to generate the data and run the benchmark
with the provided software.
}

\execSummary{

The new data economy era, based on complexly structured, distributed and large
datasets, has brought on new demands on data management and analytics.  As a
consequence, new industry actors have appeared, offering technologies specially
build for the management of graph-like data. Also, traditional database
technologies, such as relational databases, are being adapted to the new
demands to remain competitive. 

LDBC's Social Network Benchmark (LDBC-SNB) is an industry and academia
initiative, formed by principal actors in the field of graph-like data
management. His goal is to define a framework where different graph based
technologies can be fairly tested and compared, that can drive the
identification of systems' bottlenecks and required functionalities, and can
help researchers to open new research lines. 

The philosophy around which LDBC-SNB is being designed is to be easy to
understand, to be felxible and to be cheap to adopt. For all these reasons,
LDBC-SNB will propose different workloads representing all the usage scenarios
of graph-like database technologies, hence, targeting systems of different
nature and characteristics.  In order increase its adoption by industry and
reasearch institutions, LDBC-SNB provides all the necessary software, which are 
designed to be easy to use and deploy at a small cost.

This preliminary version of the LDBC-SNB specification, contains a detailed
explanation of the data used in the whole LDBC-SNB benchmark, a detailed
description for all the Interactive Workload lookup queries, and instructions
on how to generate the data and run the benchmark with the provided software.
}

\begin{document}

\maketitle

%\include{glossary}

\listoffigures
\listoftables
\chapter*{Definitions}
\input{defs}

\chapter{Introduction}
%%% INTRODUCTION %%%

%%%%%%%%%%%%%%%%%%%%%%%%%%%%%%%%%%%%%%%%%%%%%%%%%%%%%%%%%%%%%%%%%%%%%%%%%%%%%%
%%%%%%%%%%%%%%%%%%%%%%%%%%%%%%%%%%%%%%%%%%%%%%%%%%%%%%%%%%%%%%%%%%%%%%%%%%%%%%
%%%%%%%%%%%%%%%%%%%%%%%%%%%%%%%%%%%%%%%%%%%%%%%%%%%%%%%%%%%%%%%%%%%%%%%%%%%%%%

\section{Motivation for the Benchmark}

The new era of data economy, based on large, distributed and complexly
structured data sets, has brought on new and complex challenges in the field of
data management and analytics. These data sets, usually modeled as large
graphs, have attracted both the industry and academia, due to the new
opportunities in research and innovation they offer.  This situation has also
opened the door for new companies to emerge, offering new non-relational and
graph-like technologies that are called to play a significant role in upcoming
years.

The change in the data paradigm, calls for new benchmarks to test the  new
emerging technologies, as they set a framework where different systems can
compete and compare in a fair way, they let technology providers to identify
the bottlenecks and gaps of their systems and, in general, drive the research
and development of new information technology solutions. Without them, the
uptake of these technologies is at risk by not providing the industry with
clear, user-driven targets for performance and functionality.

The LDBC Social Network Benchmark (LDBC-SNB) aims at being comprehensive
benchmark setting the rules for the evaluation of graph-like data management
technologies.  LDBC-SNB is designed to be a plausible look-alike of all the
aspects of operating a social network site, as one of the most representative
and relevant use case of modern graph-like applications. LDBC-SNB is a work in
progress, and initially, it only includes the Interactive Workload,
but two more workloads will be introduced in the future: the Business
Intelligence and the Analytics. By designing three separate workloads, LDBC-SNB
targets a broader range of systems with different nature and characteristics.
LDBC-SNB aims at capturing the essential features of these usage scenarios
while abstracting away details of specific business deployments. 

%%%%%%%%%%%%%%%%%%%%%%%%%%%%%%%%%%%%%%%%%%%%%%%%%%%%%%%%%%%%%%%%%%%%%%%%%%%%%%
%%%%%%%%%%%%%%%%%%%%%%%%%%%%%%%%%%%%%%%%%%%%%%%%%%%%%%%%%%%%%%%%%%%%%%%%%%%%%%
%%%%%%%%%%%%%%%%%%%%%%%%%%%%%%%%%%%%%%%%%%%%%%%%%%%%%%%%%%%%%%%%%%%%%%%%%%%%%%

\section{Relevance to Industry}

LDBC-SNB is intended to provide the following value to different stakeholders:

\begin{itemize}
 \item For \textbf{end users} facing graph processing tasks, LDBC-SNB provides
     a recognizable scenario against which it is possible to compare merits of
     different products and technologies.  By covering a wide variety of scales
     and price points, LDBC-SNB can serve as an aid to technology selection.
 \item For \textbf{vendors} of graph database technology, LDBC-SNB provides a
     checklist of features and performance characteristics that helps in
     product positioning and can serve to guide new development.
 \item For \textbf{researchers}, both industrial and academic, the LDBC-SNB
     dataset and workload provide interesting challenges in multiple
     choke-point areas, such as query optimization, (distributed) graph
     analysis, transactional throughput, and provides a way to objectively
     compare the effectiveness and efficiency of new and existing technology in
     these areas.
\end{itemize}

The technological scope of the LDBC-SNB comprises all systems that one might
conceivably use to perform social network data management tasks:

\begin{itemize}
 \item \textbf{Graph database systems} (e.g. Neo4j, InfiniteGraph, Sparksee,
     Titan) are novel technologies aimed at storing directed and labeled
     graphs. They support graph traverals, typically by means of APIs, though
     some of them also support some sort of graph oriented query language (e.g.
     Neo4j's Cypher). These systems' internal structures are typically designed
     to store dynamic graphs that change over time.  They oftern support
     transactional queries with some degree of consistency, and value-based
     indexes to quicly locate nodes and edges. Finally, their architecture is
     typically single-machine (non-cluster). These systems can 
     potentially implement the three workloads, though Interactive and Business Intelligence
     workloads are where they will presumably be more competitive.
 \item \textbf{Graph programming frameworks} (e.g. Giraph, Signal/Collect,
     Graphlab, Green Marl) are designed to perform global graph queries
     computations, executed in parallel or lockstep. These computations are typically
     long latency, involving many nodes and edges and often consist of approximation
     answers to NP-complete problems. These systems expose an API, sometimes following
     a vertex centric paradigm, and their architecture targets both single-,machine and
     cluster systems. Though these systems will likely implement the Graph Analytics workload.
 \item \textbf{RDF database systems} (e.g. OWLIM, Virtuoso, BigData, Jena TDB,
     Stardog, Allegrograph) are systems that implement the SPARQL1.1 query
     language, similar in complexity to SQL1992, which allows for structured
     queries, and simple traversals. RDF database system often come with
     additional support for simple reasoning (sameAs,subClass), text search and
     geospatial predicates.  RDF database systems generally support
     transactions, but not always with full concurrency and serializability and
     their supposed strength is integrating multiple data sources (e.g.
     DBpedia). Their architecture is both single-machine and clustered, and
     they will likely target Interactive and Business Intelligence workloads.
 \item \textbf{Relational database systems} (e.g. Postgres, MySQL, Oracle, DB2,
     SQLserver, Virtuoso, MonetDB, Vectorwise, Vertica, but also Hive and
     Impala) treat data as relational, and queries are formulated in SQL and/or
     PL/SQL. Both single-machine and cluster systems exist.  They  do not
     normally support recursion, or stateful recursive algorithms, which makes
     them not at home in the Graph Analytics workloads
 \item \textbf{noSQL database systems} (e.g. key-value stores such as HBase,
     REDIS, MongoDB, CouchDB, or even MapReduce systems like Hadoop and Pig).
     are cluster-vbased and scalable. Key-value stores could possibly implement
     the Interactive Workload, though its navigational aspects would pose some
     problems as potentially many key-value lookups are needed. MapReduce
     systems could be suited for the Graph Analytics workload.  but their query
     latency would presumably be so high that the Business Intelligence
     workload would not make sense, though we note that some of the key-value
     stores (e.g. MongoDB) provide a MapReduce query functionality on the data
     that it stores which could make it suited for the Business Intelligence workload.
\end{itemize}

%%%%%%%%%%%%%%%%%%%%%%%%%%%%%%%%%%%%%%%%%%%%%%%%%%%%%%%%%%%%%%%%%%%%%%%%%%%%%%
%%%%%%%%%%%%%%%%%%%%%%%%%%%%%%%%%%%%%%%%%%%%%%%%%%%%%%%%%%%%%%%%%%%%%%%%%%%%%%
%%%%%%%%%%%%%%%%%%%%%%%%%%%%%%%%%%%%%%%%%%%%%%%%%%%%%%%%%%%%%%%%%%%%%%%%%%%%%%

\section{General Benchmark Overview}

LDBC-SNB aims at being a complete benchmark, designed with the following goals in mind:

\begin{itemize}
 \item \textbf{Rich coverage}. LDBC-SNB is intended to cover most demands
     encountered in the management of complexly structured data. 
 \item \textbf{Modularity}. LDBC-SNB is broken into parts that can be
     individually addressed. In this manner LDBC-SNB
     stimulates innovation without imposing an overly high threshold for
     participation.
 \item \textbf{Reasonable implementation cost}. For a product offering relevant
     functionality, the effort for obtaining initial results with SNB should be
     small, on the order of days.
 \item \textbf{Relevant selection of challenges}. Benchmarks are known to
     direct product development in certain directions. LDBC-SNB is informed by
     the state of the art in database research so as to offer optimization
     challenges for years to come while not having a prohibitively high
     threshold for entry.
 \item \textbf{Reproducibility and documentation of results}. LDBC-SNB
     will specify the rules for full disclosure of benchmark execution and for
     auditing of benchmark runs. The workloads may be run on any equipment
     but the exact configuration and price of the hardware and software must be
     disclosed.
\end{itemize}

LDBC-SNB benchmark is modeled around the operation of a real social network
site. A social network site represents a relevant use case for the following
reasons:

\begin{itemize}
    \item It is simple to understand for a large audience, as it is 
        arguably present to our every-day life in different shapes and forms.  
    \item It allows testing a complete range of interesting
        challenges, by means of different workloads targeting systems of
        different nature and characteristics.
    \item A social network can be scaled, allowing the design of a
        scalable benchmark targeting systems of different sizes and budgets.
\end{itemize}

In Section~\ref{section:data}, LDBC-SNB defines the schema of the data used in
the benchmark. The schema, represents a realistic social network, including
people and their activity in the social network during a period of time.
Personal information of each person, such as the name, the birth day, interests
or the places where people work or study, is included. Persons' activity is
represented in the form of friendhisp relationships and content sharing (i.e
messages and pictures). LDBC-SNB provides a scalable synthetic data generator
based on the MapReduce parallel paradigm, that produces networks with the
described schema with distributions and correlations similar to those expected
in a real social network. Furthermore, the data generator is designed to be
user friendly. The proposed data schema is shared by all the different proposed
workloads, those we currently have, and those that will be proposed in the future.

In Section~\ref{section:workloads}, the Interactive Workload is proposed. 
Two more workloads are planned:  Business Intelligence Workload and
Analytical workload. Workloads are designed to mimic the different usage
scenarios found in operating a real social network site, and each of them
targets one or more types of systems.  Each workload defines a set of queries
and query mixes, designed to stress the SUTs in different choke-point areas,
while being credible and realistic. Interactive workload reproduces the
interaction between the users of the social network by including lookups and
transactions that update small portions of the data base.  These queries are
designed to be interactive and target systems capable of responding such
queries with low latency for multiple concurrent users. Business Intelligence
workload, will represent those business intelligence analytics a social
network company would like to perform in the social network, in order to take
advantage of the data to discover new business opportunities. This workload
will explore moderate portions of data from different entities, and performing more
resource intensive operations. Finally, the graph analytics workload will aim at
exploring the characteristics of the underlying structure of the network. Shortest
paths, community detection or centrality, are representative queries of this workload,
and will imply touching a vast amount of the dataset.
 

LDBC-SNB provides an execution test driver, which is responsible of executing
the workloads and gathering the results. The driver is designed with simplicity
and portability in mind, to ease the implementation on systems with different
nature and characteristics, at a low implementation cost. Furthermore, it
automatically handles the validation of the queries by means of a validation
dataset provided by LDBC.  The overall philosophy of LDBC-SNB is to provide all
the necessary software tools to run the benchmark, and therefore to reduce the
benchmark's entry point as much as possible.

Detailed instructions to generate the required datasets and to run Interactive
Workload of the benchmark, are described in Chapter~\ref{chapter:instructions}.
Finally, in the Appendix, Interactive Workload query implementation examples in
Virtuoso's SQL, Virtuoso's SPARQL and Neo4j Cypher are shown.


\section{Participation of Industry and Academia}

The list of institutions that take part in the definition and development
of LDBC-SNB is formed by relevant actors from both the industry and academia in
the field of linked data management. All the participants have contributed with
their experience and expertise in the field, making a credible and relevant
benchmark that meets all the desired needs. The list of participants is the
following:

\begin{itemize}
    \item FOUNDATION FOR RESEARCH AND TECHNOLOGY HELLAS
    \item NEO4J
    \item ONTOTEXT
    \item OPENLINK
    \item TECHNISCHE UNIVERSITAET MUENCHEN
    \item UNIVERSITAET INNSBRUCK
    \item UNIVERSITAT POLITECNICA DE CATALUNYA
    \item VRIJE UNIVERSITEIT AMSTERDAM
\end{itemize}

\begin{figure}
\end{figure}

Besides the aforementioned institutions, during the development of the
benchmark several meetings with the technical and users community have been
conducted, receiving an invaluable feedback that has contributed to the whole
development of the benchmark in every of its aspects.

%%%%%%%%%%%%%%%%%%%%%%%%%%%%%%%%%%%%%%%%%%%%%%%%%%%%%%%%%%%%%%%%%%%%%%%%%%%%%%
%%%%%%%%%%%%%%%%%%%%%%%%%%%%%%%%%%%%%%%%%%%%%%%%%%%%%%%%%%%%%%%%%%%%%%%%%%%%%%
%%%%%%%%%%%%%%%%%%%%%%%%%%%%%%%%%%%%%%%%%%%%%%%%%%%%%%%%%%%%%%%%%%%%%%%%%%%%%%



\chapter{Formal Definition}
\input{definition}

\chapter{Implementation Instructions}\label{chapter:instructions}
\input{instructions}

\bibliographystyle{plain}
\bibliography{references}

\appendix

\chapter{Interactive Query Set Implementations}

\section{Virtuoso SPARQL 1.1}

\subsection{Query 1} 
{\footnotesize
\begin{verbatim}
 sparql select ?fr ?last min(?dist) as ?mindist  ?bday ?since ?gen ?browser ?locationIP 
    ((select group_concat (?email, ", ")
        where {
            ?frr snvoc:email ?email .
            filter (?frr = ?fr) .
        }
        group by ?frr)) as ?email
    ((select group_concat (?lng, ", ")
        where {
            ?frr snvoc:speaks ?lng .
            filter (?frr = ?fr) .
        }
        group by ?frr)) as ?lng
    ?based
    ((select group_concat ( bif:concat (?o_name, " ", ?year, " ", ?o_country), ", ")
        where {
        ?frr snvoc:studyAt ?w .
        ?w snvoc:classYear ?year .
            ?w snvoc:hasOrganisation ?org .
            ?org snvoc:isLocatedIn ?o_countryURI .
        ?o_countryURI foaf:name ?o_country .
            ?org foaf:name ?o_name .
        filter (?frr = ?fr) .
        }
        group by ?frr)) as ?studyAt
    ((select group_concat ( bif:concat (?o_name, " ", ?year, " ", ?o_country), ", ")
        where {
            ?frr snvoc:workAt ?w .
            ?w snvoc:workFrom ?year .
            ?w snvoc:hasOrganisation ?org .
            ?org snvoc:isLocatedIn ?o_countryURI .
            ?o_countryURI foaf:name ?o_country .
            ?org foaf:name ?o_name .
            filter (?frr = ?fr) .
        }
        group by ?frr)) as ?workAt
    {
        ?fr a snvoc:Person .
        ?fr snvoc:firstName "%Name%" .
        ?fr snvoc:lastName ?last .
        ?fr snvoc:birthday ?bday .
        ?fr snvoc:isLocatedIn ?basedURI .
        ?basedURI foaf:name ?based .
        ?fr snvoc:creationDate ?since .
        ?fr snvoc:gender ?gen .
        ?fr snvoc:locationIP ?locationIP .
        ?fr snvoc:browserUsed ?browser .

        {
          { select distinct ?fr (1 as ?dist)
            where {
              sn:pers%Person% snvoc:knows ?fr.
            }
          }
      union
          { select distinct ?fr (2 as ?dist)
            where {
              sn:pers%Person% snvoc:knows ?fr2. 
                              ?fr2 snvoc:knows ?fr.
                              filter (?fr != sn:pers%Person%).
            }
          }
      union
          { select distinct ?fr (3 as ?dist)
            where {
              sn:pers%Person% snvoc:knows ?fr2. 
                              ?fr2 snvoc:knows ?fr3. 
                              ?fr3 snvoc:knows ?fr. 
                              filter (?fr != sn:pers%Person%).
            }
          } .
        }
    }
    group by ?fr ?last ?bday ?since ?gen ?browser ?locationIP ?based
    order by ?mindist ?last ?fr
    limit 20
\end{verbatim}
}
 

\subsection{Query 2}

{\footnotesize
\begin{verbatim}
 sparql select ?fr ?first ?last ?post ?content ?date 
from <sib>
where {
  sn:pers%Person% snvoc:knows ?fr.
  ?fr snvoc:firstName ?first. ?fr snvoc:lastName ?last .
  ?post snvoc:hasCreator ?fr.
  { {?post snvoc:content ?content } union { ?post snvoc:imageFile ?content }} .
  ?post snvoc:creationDate ?date.
  filter (?date <= "%Date0%"^^xsd:date).
}
order by desc (?date) ?post
limit 20
\end{verbatim}
}


\subsection{Query 3}

{\footnotesize
\begin{verbatim}
 sparql select ?fr ?first ?last ?ct1 ?ct2 (?ct1 + ?ct2) as ?sum 
from <sib>  
where { 
    {select distinct ?fr ?first ?last
        (((select count (*)
        where {
            ?post snvoc:hasCreator ?fr .
            ?post snvoc:creationDate ?date .
            filter (?date >= "%Date0%"^^xsd:date && 
                    ?date < bif:dateadd ("day", %Duration%, "%Date0%"^^xsd:date)) .
            ?post snvoc:isLocatedIn dbpedia:%Country1%
        }))
        as ?ct1)
        ((select count (*)
        where {
            ?post2 snvoc:hasCreator ?fr .
            ?post2 snvoc:creationDate ?date2 . 
            filter (?date2 >= "%Date0%"^^xsd:date && 
                    ?date2 < bif:dateadd ("day", %Duration%, "%Date0%"^^xsd:date)) .
            ?post2 snvoc:isLocatedIn dbpedia:%Country2%
        })
        as ?ct2)
    where {
        {sn:pers%Person% snvoc:knows ?fr.} union { sn:pers%Person% snvoc:knows ?fr2.
                                                    ?fr2 snvoc:knows ?fr.
                                                    filter (?fr != sn:pers%Person%)
                                                }.
        ?fr snvoc:firstName ?first . ?fr snvoc:lastName ?last .
        ?fr snvoc:isLocatedIn ?city .
    filter(!exists {?city snvoc:isPartOf dbpedia:%Country1%}).
    filter(!exists {?city snvoc:isPartOf dbpedia:%Country2%}).
    }
    }.
    filter (?ct1 > 0 && ?ct2 > 0) .
}
order by desc(6) ?fr
limit 20
\end{verbatim}
}

 

\subsection{Query 4}
{\footnotesize
\begin{verbatim}
sparql select ?tagname count (*) #Q4
from <sib>
where {
    ?post a snvoc:Post .
    ?post snvoc:hasCreator ?fr .
    ?post snvoc:hasTag ?tag .
    ?tag foaf:name ?tagname .
    ?post snvoc:creationDate ?date . 
    sn:pers%Person% snvoc:knows ?fr .
    filter (?date >= "%Date0%"^^xsd:date && 
            ?date <= bif:dateadd ("day", %Duration%, "%Date0%"^^xsd:date) ) .
    filter (!exists {
        sn:pers%Person% snvoc:knows ?fr2 .
        ?post2 snvoc:hasCreator ?fr2 .
        ?post2 snvoc:hasTag ?tag .
        ?post2 snvoc:creationDate ?date2 .
        filter (?date2 < "%Date0%"^^xsd:date)}) 
    }
group by ?tagname
order by desc(2) ?tagname
limit 10
\end{verbatim}
}


\subsection{Query 5}
{\footnotesize
\begin{verbatim}
sparql select ?title count (*) #Q5
from <sib>
where {
    {select distinct ?fr
        from <sib>
        where {
            {sn:pers%Person% snvoc:knows ?fr.} 
            union {sn:pers%Person% snvoc:knows ?fr2. 
                   ?fr2 snvoc:knows ?fr. 
                   filter (?fr != sn:pers%Person%)}
        }
    } .
    ?group snvoc:hasMember ?mem .
    ?mem snvoc:hasPerson ?fr .
    ?mem snvoc:joinDate ?date .
    filter (?date >= "%Date0%"^^xsd:date) .
    ?post snvoc:hasCreator ?fr .
    ?group snvoc:containerOf ?post .
    ?group snvoc:title ?title.
}
group by ?title
order by desc(2) ?title
limit 20
\end{verbatim}
}
 

\subsection{Query 6}
{\footnotesize
\begin{verbatim}
sparql select ?tagname count (*) 
from <sib>
where {  
    {  select distinct ?fr
       from <sib>
       where {
           {sn:pers%Person% snvoc:knows ?fr.} union { sn:pers%Person% snvoc:knows ?fr2.
                                                      ?fr2 snvoc:knows ?fr.
                                                      filter (?fr != sn:pers%Person%) }
       }
    } .
    ?post snvoc:hasCreator ?fr .
    ?post snvoc:hasTag ?tag1 .
    ?tag1 foaf:name ?tagname1 .
    filter (?tagname1 != '%Tag%') .
    ?post snvoc:hasTag ?tag .
    ?tag foaf:name ?tagname .
}
group by ?tagname
order by desc(2) ?tagname
limit 10
\end{verbatim}
}




\subsection{Query 7}
{\footnotesize
\begin{verbatim}
 sparql select ?liker ?first ?last ?ldt 
       (if ((exists {  sn:pers%Person% snvoc:knows ?liker}), 0, 1) as ?is_new)
       ?post ?content (bif:datediff ("minute", ?dt, ?ldt) as ?lag) 
from <sib>
where {
  ?post snvoc:hasCreator sn:pers%Person% .
  {{ ?post snvoc:content ?content } union {?post snvoc:imageFile ?content}} .
  ?lk snvoc:hasPost ?post .
  ?liker snvoc:likes ?lk . ?liker  snvoc:firstName ?first . ?liker snvoc:lastName ?last . 
  ?post snvoc:creationDate ?dt . ?lk snvoc:creationDate ?ldt .
}
order by desc (?ldt) ?liker
limit 20
\end{verbatim}
}

 

\subsection{Query 8}

{\footnotesize
\begin{verbatim}
sparql select ?from ?first ?last ?dt ?rep ?content 
where {
  { select ?rep ?dt
    where {
        ?post snvoc:hasCreator sn:pers%Person% .
        ?rep snvoc:replyOf ?post . ?rep snvoc:creationDate ?dt .
    }
    order by desc (?dt)
    limit 20
  } .
  ?rep snvoc:hasCreator ?from .
  ?from snvoc:firstName ?first . ?from snvoc:lastName ?last . 
  ?rep snvoc:content ?content.
}
order by desc(?dt) ?rep
\end{verbatim}
}

\subsection{Query 9}

{\footnotesize
\begin{verbatim}
sparql select ?fr ?first ?last ?post ?content ?date 
from <sib>
where {
  {select distinct ?fr
   from <sib>
   where {
       {sn:pers%Person% snvoc:knows ?fr.} union { sn:pers%Person% snvoc:knows ?fr2.
                                                  ?fr2 snvoc:knows ?fr.
                                                  filter (?fr != sn:pers%Person%) }
   }
  }
  ?fr snvoc:firstName ?first . ?fr snvoc:lastName ?last .
  ?post snvoc:hasCreator ?fr.
  ?post snvoc:creationDate ?date.
  filter (?date < "%Date0%"^^xsd:date).
  {{?post snvoc:content ?content} union {?post snvoc:imageFile ?content}} .
}
order by desc (?date) ?post
limit 20
\end{verbatim}
}

\subsection{Query 10}
{\footnotesize
\begin{verbatim}
 sparql select ?first ?last 
    ((( select  count (distinct ?post)
        where {
            ?post snvoc:hasCreator ?fof .
            ?post snvoc:hasTag ?tag .
            sn:pers%Person% snvoc:hasInterest ?tag
        }
    ))
    -
    ((  select  count (distinct ?post)
        where {
            ?post snvoc:hasCreator ?fof .
            ?post snvoc:hasTag ?tag .
            filter (!exists {sn:pers%Person% snvoc:hasInterest ?tag})
        }
    )) as ?score)
    ?fof  ?gender ?locationname
from <sib>
where {
   {select distinct ?fof
    where {
        sn:pers%Person% snvoc:knows ?fr .
        ?fr snvoc:knows ?fof .
    filter (?fof != sn:pers%Person%)
        minus { sn:pers%Person% snvoc:knows ?fof } .
    }
   } .
   ?fof snvoc:firstName ?first .
   ?fof snvoc:lastName ?last .
   ?fof snvoc:gender ?gender .
   ?fof snvoc:birthday ?bday .
   ?fof snvoc:isLocatedIn ?based .
   ?based foaf:name ?locationname .
   filter (1 = if (bif:month (?bday) = %HS0%, if (bif:dayofmonth (?bday) > 21, 1, 0),
               if (bif:month (?bday) = %HS1%, if (bif:dayofmonth(?bday) < 22, 1, 0), 0)))
}
order by desc(3) ?fof
limit 10
\end{verbatim}
}

 

\subsection{Query 11}
{\footnotesize
\begin{verbatim}
 sparql select ?first ?last ?startdate ?orgname ?fr 
where {
    ?w snvoc:hasOrganisation ?org .
    ?org foaf:name ?orgname .
    ?org snvoc:isLocatedIn ?country.
    ?country foaf:name '%Country%' .
    ?fr snvoc:workAt ?w .
    ?w snvoc:workFrom ?startdate .
    filter (?startdate < %Date0%) .
    {  select distinct ?fr
       from <sib>
       where {
           {sn:pers%Person% snvoc:knows ?fr.} union { sn:pers%Person% snvoc:knows ?fr2.
                                                      ?fr2 snvoc:knows ?fr.
                                                      filter (?fr != sn:pers%Person%) }
       }
    } .
    ?fr snvoc:firstName ?first .
    ?fr snvoc:lastName ?last .
}
order by ?startdate ?fr ?orgname
limit 10
\end{verbatim}
}


\subsection{Query 12}
{\footnotesize
\begin{verbatim}
sparql select ?exp ?first ?last sql:group_concat_distinct(?tagname) count (*) #Q12
where {
    sn:pers%Person% snvoc:knows ?exp .
    ?exp snvoc:firstName ?first . ?exp snvoc:lastName ?last .
    ?reply snvoc:hasCreator ?exp .
    ?reply snvoc:replyOf  ?org_post .
    filter (!exists {?org_post snvoc:replyOf ?xx}) .
    ?org_post snvoc:hasTag ?tag .
    ?tag foaf:name ?tagname .
    ?tag a ?type.
    ?type rdfs:subClassOf* ?type1 .
    ?type1 rdfs:label %TagType% .
}
group by ?exp ?first ?last
order by desc(5) ?exp
limit 20
\end{verbatim}
}


\subsection{Query 13} 
{\footnotesize
\begin{verbatim}
sparql select count(*) 
where
  {
    {
      select ?s ?o
      where
        {
          ?s snvoc:knows ?o.
        }
    }
    option ( transitive,
             t_distinct,
             t_in(?s),
             t_out(?o),
             t_shortest_only,
             t_direction 3,
             t_step ('path_id') as ?path_no) .
    filter ( ?s = sn:pers%Person1% ).
    filter ( ?o = sn:pers%Person2% ).
    filter (?path_no = 0).
  }
\end{verbatim}
}

\subsection{Query 14}
{\footnotesize
\begin{verbatim}
create procedure path_str_sparql (in path any)
{
  declare str any;
  declare inx int;
  str := '';
  foreach (any  st  in path) do
    str := str || sprintf (' %d->%d (%d) ',
                            cast (substring(st[0], 48, 20) as int), 
                            coalesce(cast (substring(st[1], 48, 20) as int), 0),
                            coalesce (st[2], 0));
  return str;
}

create procedure c_weight_sparql (in p1 varchar, in p2 varchar)
{
  vectored;
  if (p1 is null or p2 is null)
     return 0;
  return 0.5 + 
       ( sparql select count(*) from <sib> where {?post1 snvoc:hasCreator ?:p1.
                                                  ?post1 snvoc:replyOf ?post2.
                                                  ?post2 snvoc:hasCreator ?:p2.
                                                  ?post2 a snvoc:Post} ) +
       ( sparql select count(*) from <sib> where {?post1 snvoc:hasCreator ?:p2.
                                                  ?post1 snvoc:replyOf ?post2.
                                                  ?post2 snvoc:hasCreator ?:p1.
                                                  ?post2 a snvoc:Post} ) +
       ( sparql select 0.5 * count(*) from <sib> where {?post1 snvoc:hasCreator ?:p1.
                                                        ?post1 snvoc:replyOf ?post2. 
                                                        ?post2 snvoc:hasCreator ?:p2.
                                                        ?post2 a snvoc:Comment} ) +
       ( sparql select 0.5 * count(*) from <sib> where {?post1 snvoc:hasCreator ?:p2.
                                                        ?post1 snvoc:replyOf ?post2.
                                                        ?post2 snvoc:hasCreator ?:p1.
                                                        ?post2 a snvoc:Comment} );
}

select sql:path_str_sparql(?path), ?sc
where
{
  select ?path_no, sql:vector_agg (bif:vector (?via1, ?via2, ?cweight))
                                    as ?path, sum (?cweight) 
                                    as ?sc
  where
  {
    select ?via1 ?via2 ?path_no ?step_no sql:c_weight_sparql(?via1, ?via2) as ?cweight
    where
    {
      {
        select ?s bif:idn(?s) as ?via2 ?o
        where
        {
          ?s snvoc:knows ?o1.
      ?o1 snvoc:hasPerson ?o .
        }
      }
      option ( transitive,
             t_distinct,
             t_in(?s),
             t_out(?o),
         t_shortest_only,
         t_direction 3,
         t_step (?s) as ?via1,
         t_step ('path_id') as ?path_no,
             t_step ('step_no') as ?step_no ) .
      filter ( ?s = %Person1% ).
      filter ( ?o = %Person2% ).
    }
  }
  group by ?path_no
}
order by desc(?sc)
limit 10
\end{verbatim}
}


\section{Virtuoso SQL}
Important: Virtuoso SQL implementation assumes that both Post and Comment entities share the same table.
\subsection{Query 1}

{\footnotesize
\begin{verbatim}
select top 20 id, p_lastname, min (dist) as dist,
       p_birthday, p_creationdate, p_gender, p_browserused,
       bit_shift(bit_and(p_locationip, 4278190080), -24) || '.' ||
       bit_shift(bit_and(p_locationip, 16711680), -16) || '.' ||
       bit_shift(bit_and(p_locationip, 65280), -8) || '.' ||
       bit_and(p_locationip, 255) as ip,
       (select group_concat (pe_email, ', ') 
            from person_email 
            where pe_personid = id 
            group by pe_personid) as emails,
       (select group_concat (plang_language, ', ') 
            from person_language
            where plang_personid = id
            group by plang_personid) as languages,
       p1.pl_name,
       (select group_concat (o2.o_name || ' ' || pu_classyear || ' ' || p2.pl_name, ', ') 
                from person_university, organisation o2, place p2  
                where pu_personid = id and 
                      pu_organisationid = o2.o_organisationid and
                      o2.o_placeid = p2.pl_placeid 
                group by pu_personid) as university,
       (select group_concat (o3.o_name || ' ' || pc_workfrom || ' ' || p3.pl_name, ', ') 
                from person_company, organisation o3, place p3
                where pc_personid = id and 
                      pc_organisationid = o3.o_organisationid and 
                      o3.o_placeid = p3.pl_placeid 
                group by pc_personid) as company
from
    (
    select k_person2id as id, 1 as dist from knows, person 
                                        where k_person1id = @Person@ and 
                                              p_personid = k_person2id and 
                                              p_firstname = '@Name@'
    union all
    select b.k_person2id as id, 2 as dist from knows a, knows b, person
    where
      a.k_person1id = @Person@ and 
      b.k_person1id = a.k_person2id and 
      p_personid = b.k_person2id and 
      p_firstname = '@Name@'
    union all
    select c.k_person2id as id, 3 as dist from knows a, knows b, knows c, person
    where
      a.k_person1id = @Person@ and 
      b.k_person1id = a.k_person2id and 
      b.k_person2id = c.k_person1id and 
      p_personid = c.k_person2id and
      p_firstname = '@Name@'
    ) tmp, person, place p1
  where
    p_personid = id and
    p_placeid = p1.pl_placeid
  group by id, p_lastname
  order by dist, p_lastname, id
\end{verbatim}
}
 

\subsection{Query 2}
{\footnotesize
\begin{verbatim}
select top 20 p_personid as personid, p_firstname as firstname, p_lastname as lastname,
       ps_postid as id, ps_content || ps_imagefile as content, ps_creationdate as creationdate
from person, post, knows
where
    p_personid = ps_creatorid and
    ps_creationdate <= stringdate('@Date0@') and
    k_person1id = @Person@ and
    k_person2id = p_personid
order by creationdate desc, id
\end{verbatim}
}

\subsection{Query 3}

{\footnotesize
\begin{verbatim}
select top 20 p_personid, p_firstname, p_lastname, ct1, ct2, total
from
 ( select k_person2id
   from knows
   where
   k_person1id = @Person@
   union
   select k2.k_person2id
   from knows k1, knows k2
   where
   k1.k_person1id = @Person@ and 
   k1.k_person2id = k2.k_person1id and 
   k2.k_person2id <> @Person@
 ) f,  person, place p1, place p2,
 (
  select chn.ps_c_creatorid, ct1, ct2, ct1 + ct2 as total
  from
   (
      select ps_creatorid as ps_c_creatorid, count(*) as ct1 from post, place
      where
        ps_locationid = pl_placeid and 
        pl_name = '@Country1@' and
        ps_creationdate between 
            stringdate('@Date0@') and dateadd ('day', @Duration@, stringdate('@Date0@'))
      group by ps_c_creatorid
   ) chn,
   (
      select ps_creatorid as ps_c_creatorid, count(*) as ct2 from post, place
      where
        ps_locationid = pl_placeid and 
        pl_name = '@Country2@' and
        ps_creationdate between 
            stringdate('@Date0@') and 
            dateadd ('day', @Duration@, stringdate('@Date0@'))
      group by ps_c_creatorid
   ) ind
  where CHN.ps_c_creatorid = IND.ps_c_creatorid
 ) cpc
where
f.k_person2id = p_personid and p_placeid = p1.pl_placeid and
p1.pl_containerplaceid = p2.pl_placeid and 
p2.pl_name <> '@Country1@' and 
p2.pl_name <> '@Country2@' and
f.k_person2id = cpc.ps_c_creatorid
order by 6 desc, 1
\end{verbatim}
}
 

\subsection{Query 4}
{\footnotesize
\begin{verbatim}
select top 10 t_name, count(*)
from tag, post, post_tag, knows
where
    ps_postid = pst_postid and
    pst_tagid = t_tagid and
    ps_creatorid = k_person2id and
    k_person1id = @Person@ and
    ps_creationdate between stringdate('@Date0@') and 
                            dateadd ('day', @Duration@, stringdate('@Date0@')) and
                            isnull(ps_replyof) and
    not exists (
        select * from post, post_tag, knows
        where
        k_person1id = @Person@ and
        k_person2id = ps_creatorid and
        pst_postid = ps_postid and
        pst_tagid = t_tagid and
        ps_creationdate < '@Date0@'
    )
group by t_name
order by 2 desc, t_name
\end{verbatim}
}


\subsection{Query 5}
{\footnotesize
\begin{verbatim}
select top 20 f_title, count(*)
from forum, post, forum_person,
 ( select k_person2id
   from knows
   where
   k_person1id = @Person@
   union
   select k2.k_person2id
   from knows k1, knows k2
   where
   k1.k_person1id = @Person@ and 
   k1.k_person2id = k2.k_person1id and
   k2.k_person2id <> @Person@
 ) f
where f_forumid = ps_forumid and 
      f_forumid = fp_forumid and 
      fp_personid = f.k_person2id and 
      ps_creatorid = f.k_person2id and
      fp_creationdate >= stringdate('@Date0@')
group by f_title
order by 2 desc, f_title
\end{verbatim}
}

\subsection{Query 6}
{\footnotesize
\begin{verbatim}
select top 10 t_name, count(*)
from tag, post_tag, post,
 ( select k_person2id
   from knows
   where
   k_person1id = @Person@
   union
   select k2.k_person2id
   from knows k1, knows k2
   where
   k1.k_person1id = @Person@ and 
   k1.k_person2id = k2.k_person1id and 
   k2.k_person2id <> @Person@
 ) f
where
        isnull(ps_replyof) and
ps_creatorid = f.k_person2id and
ps_postid = pst_postid and
pst_tagid = t_tagid and
t_name <> '@Tag@' and
exists (select * from tag, post_tag where pst_postid = ps_postid and 
                                          pst_tagid = t_tagid and 
                                          t_name = '@Tag@')
group by t_name
order by 2 desc, t_name
\end{verbatim}
}


\subsection{Query 7}
{\footnotesize
\begin{verbatim}
select top 20 p_personid , p_firstname, p_lastname, l_creationdate,
              (case when k_person2id is null then 1 else 0 end) as is_new,
              ps_postid, content, lag
from
(select p_personid, p_firstname, p_lastname, l_creationdate,
        ps_postid, ps_content || ps_imagefile as content,
    datediff('minute', ps_creationdate, l_creationdate) as lag
from likes, post, person
where
    p_personid = l_personid and
    ps_postid = l_postid and
    ps_creatorid = @Person@
) p
left join
(select * from knows where k_person1id = @Person@) k
on k.k_person2id = p.p_personid
order by l_creationdate desc, 1
\end{verbatim}
}
 

\subsection{Query 8}
{\footnotesize
\begin{verbatim}
select top 20 p1.ps_creatorid, 
              p_firstname, 
              p_lastname, 
              p1.ps_creationdate, 
              p1.ps_postid, 
              p1.ps_content
  from post p1, post p2, person
  where
      p1.ps_replyof = p2.ps_postid and
      p2.ps_creatorid = @Person@ and
      p_personid = p1.ps_creatorid
order by p1.ps_creationdate desc, 5
\end{verbatim}
}



\subsection{Query 9}
{\footnotesize
\begin{verbatim}
select top 20 p_personid, p_firstname, p_lastname,
       ps_postid, ps_content || ps_imagefile, ps_creationdate
from person, post,
  ( select k_person2id
    from knows
    where
    k_person1id = @Person@
    union
    select k2.k_person2id
    from knows k1, knows k2
    where k1.k_person1id = @Person@ and
          k1.k_person2id = k2.k_person1id and 
          k2.k_person2id <> @Person@
  ) f
where
  p_personid = ps_creatorid and p_personid = f.k_person2id and
  ps_creationdate < stringdate('@Date0@')
order by ps_creationdate desc, 4
\end{verbatim}
}
 

\subsection{Query 10}
{\footnotesize
\begin{verbatim}
select top 10 p_firstname, p_lastname,
       ( select count(distinct ps_postid)
         from post, post_tag pt1
         where ps_creatorid = p_personid and
               ps_postid = pst_postid and
     exists (select * from person_tag
                      where pt_personid = @Person@ and 
                            pt_tagid = pt1.pst_tagid)
       ) -
       ( select count(distinct ps_postid)
         from post, post_tag pt1
         where ps_creatorid = p_personid and 
               ps_postid = pst_postid and
     not exists (select * from person_tag 
                          where pt_personid = @Person@ and 
                                pt_tagid = pt1.pst_tagid)
       ) as score,
       p_personid, p_gender, pl_name
from person, place,
 ( select distinct k2.k_person2id
   from knows k1, knows k2
   where k1.k_person1id = @Person@ and 
         k1.k_person2id = k2.k_person1id and 
         k2.k_person2id <> @Person@ and
   not exists (select * from knows 
                        where k_person1id = @Person@ and 
                              k_person2id = k2.k_person2id)
 ) f
where
p_placeid = pl_placeid and
p_personid = f.k_person2id and
case month(p_birthday) 
    when @HS0@ then (case when dayofmonth(p_birthday) > 21 then 1 else 0 end)
    when @HS1@ then (case when dayofmonth(p_birthday) < 22 then 1 else 0 end)
    else 0
end
order by 3 desc, 4
\end{verbatim}
}


 

\subsection{Query 11}

{\footnotesize
\begin{verbatim}
select top 10 p_firstname, p_lastname, pc_workfrom, o_name, p_personid
from person, person_company, organisation, place,
 ( select k_person2id
   from knows
   where
   k_person1id = @Person@
   union
   select k2.k_person2id
   from knows k1, knows k2
   where k1.k_person1id = @Person@ and 
         k1.k_person2id = k2.k_person1id and 
         k2.k_person2id <> @Person@
 ) f
where
    p_personid = f.k_person2id and
    p_personid = pc_personid and
    pc_organisationid = o_organisationid and
    pc_workfrom < @Date0@ and
    o_placeid = pl_placeid and
    pl_name = '@Country@'
order by pc_workfrom, 5, o_name
\end{verbatim}
}


\subsection{Query 12}
{\footnotesize
\begin{verbatim}
select top 20 p_personid, 
              p_firstname, 
              p_lastname, 
              group_concat_distinct(t_name, ', '), count(*)
from person, post p1, 
             knows, 
             post p2, 
             post_tag, 
             tag, 
             tag_tagclass
where
  k_person1id = @Person@ and
  k_person2id = p_personid and
  p_personid = p1.ps_creatorid and
  p1.ps_replyof = p2.ps_postid and
  p2.ps_replyof is null and
  p2.ps_postid = pst_postid and
  pst_tagid = t_tagid and
  t_tagid = ttc_tagid and
  (ttc_tagclassid in (
           select s_subtagclassid from
             (select transitive t_in (1) 
                                t_out (2) 
                                t_distinct 
                                s_subtagclassid, s_supertagclassid
             from subclass) k, tagclass
         where tc_tagclassid = k.s_supertagclassid and tc_name = '@TagType@'
         )
   or
   ttc_tagclassid = (select tc_tagclassid from tagclass where tc_name = '@TagType@')
   )
group by 1, p_firstname, p_lastname
order by 5 desc, 1

\end{verbatim}
}


\subsection{Query 13}
{\footnotesize
\begin{verbatim}
select count(*)
from
  (select transitive t_in (1) 
                     t_out (2) 
                     t_distinct 
                     t_shortest_only 
                     t_direction 3
   k_person1id as p1, k_person2id as p2, t_step ('path_id') as path_no from knows) kt
where
  p1 = @Person1@ and
  p2 = @Person2@ and
  path_no = 0
\end{verbatim}
}


\subsection{Query 14}
{\footnotesize
\begin{verbatim}
create procedure path_str (in path any)
{
  declare str any;
  declare inx int;
  str := '';
  foreach (any  st  in path) do
    str := str || sprintf (' %d->%d (%d) ', st[0], coalesce (st[1], 0), coalesce (st[2], 0));
  return str;
} 
create procedure c_weight (in p1 bigint, in p2 bigint)
{
  vectored;
  if (p1 is null or p2 is null)
     return 0;
  return 0.5 +
       (select count (*) 
            from post ps1, post ps2
            where ps1.ps_creatorid = p1 and 
                  ps1.ps_replyof = ps2.ps_postid and
                  ps2.ps_creatorid = p2 and 
                  ps2.ps_replyof is null) +
       (select count (*) from post ps1, post ps2
            where ps1.ps_creatorid = p2 and 
                  ps1.ps_replyof = ps2.ps_postid and
                  ps2.ps_creatorid = p1 and 
                  ps2.ps_replyof is null) +
       (select 0.5 * count (*) 
            from post c1, post c2
            where c1.ps_creatorid = p1 and
                  c1.ps_replyof = c2.ps_postid and
                  c2.ps_creatorid = p2 and 
                  c2.ps_replyof is not null) +
       (select 0.5 * count (*) 
            from post c1, post c2
            where c1.ps_creatorid = p2 and 
                  c1.ps_replyof = c2.ps_postid and 
                  c2.ps_creatorid = p1 and 
                  c2.ps_replyof is not null);
} 
select top 10 path_str (path), sc
from
  (select path_no, vector_agg (vector (via1, via2, cweight)) as path, sum (cweight) as sc
   from
       (select path_no, step_no, via1, via2, c_weight (via1, via2) as cweight
        from
          (select transitive t_in (1) 
                             t_out (2) 
                             t_distinct 
                             t_shortest_only 
                             t_direction 3
                  k_person1id as p1, 
                  k_person2id as p2, 
                  t_step (1) as via1, idn (k_person1id) as via2,
                  t_step ('path_id') as path_no, t_step ('step_no') as step_no from knows) kt
        where p1 = @Person1@ and p2 = @Person2@) w
   group by path_no) paths
order by sc desc
\end{verbatim}
}


\section{Neo API}

Due to the verbosity of the Neo4j API implementations, they are not included here but uploaded into the
following repository~\cite{neo-api}.

\section{Neo Cypher}

\subsection{Query 1}
{\footnotesize
\begin{verbatim}
MATCH (:Person {id:{1}})-[path:KNOWS*1..3]-(friend:Person)
WHERE friend.firstName = {2}
WITH friend, min(length(path)) AS distance
ORDER BY distance ASC, friend.lastName ASC, friend.id ASC
LIMIT {3}
MATCH (friend)-[:IS_LOCATED_IN]->(friendCity:City)
OPTIONAL MATCH (friend)-[studyAt:STUDY_AT]->(uni:University)-[:IS_LOCATED_IN]->(uniCity:City)
WITH friend, 
  collect(CASE uni.name 
            WHEN null THEN null 
            ELSE [uni.name, studyAt.classYear, uniCity.name] 
          END) AS unis, 
  friendCity, 
  distance
OPTIONAL MATCH (friend)-[worksAt:WORKS_AT]->(company:Company)-[:IS_LOCATED_IN]->(companyCountry:Country)
WITH friend, 
  collect(CASE company.name 
            WHEN null THEN null 
            ELSE [company.name, worksAt.workFrom, companyCountry.name] 
          END) AS companies, 
  unis, 
  friendCity, 
  distance
RETURN 
  friend.id AS id, 
  friend.lastName AS lastName, 
  distance, 
  friend.birthday AS birthday, 
  friend.creationDate AS creationDate, 
  friend.gender AS gender, 
  friend.browserUsed AS browser, 
  friend.locationIP AS locationIp, 
  friend.email AS emails, 
  friend.languages AS languages, 
  friendCity.name AS cityName, 
  unis, 
  companies
ORDER BY distance ASC, friend.lastName ASC, friend.id ASC
LIMIT {3}
\end{verbatim}
}

\subsection{Query 2}

{\footnotesize
\begin{verbatim}
MATCH (:Person {id:{1}})-[:KNOWS]-(friend:Person)<-[:HAS_CREATOR]-(message)
WHERE message.creationDate <= {2} AND (message:Post OR message:Comment)
RETURN 
  friend.id AS personId, 
  friend.firstName AS personFirstName, 
  friend.lastName AS personLastName, 
  message.id AS messageId, 
  CASE has(message.content) 
    WHEN true THEN message.content 
    ELSE message.imageFile 
  END AS messageContent,
  message.creationDate AS messageDate
ORDER BY messageDate DESC, messageId ASC
LIMIT {3}
\end{verbatim}
}

\subsection{Query 3}

{\footnotesize
\begin{verbatim}
MATCH (person:Person {id:{1}})-[:KNOWS*1..2]-(friend:Person)<-[:HAS_CREATOR]-(messageX),
      (messageX)-[:IS_LOCATED_IN]->(countryX:Country)
WHERE not(person=friend) AND 
  not((friend)-[:IS_LOCATED_IN]->()-[:IS_PART_OF]->(countryX)) 
  AND countryX.name={2} 
  AND messageX.creationDate>={4} 
  AND messageX.creationDate<{5}
WITH friend, count(DISTINCT messageX) AS xCount
MATCH (friend)<-[:HAS_CREATOR]-(messageY)-[:IS_LOCATED_IN]->(countryY:Country)
WHERE countryY.name={3} 
  AND not((friend)-[:IS_LOCATED_IN]->()-[:IS_PART_OF]->(countryY)) 
  AND messageY.creationDate>={4} 
  AND messageY.creationDate<{5}
WITH friend.id AS friendId, 
  friend.firstName AS friendFirstName, 
  friend.lastName AS friendLastName, 
  xCount, 
  count(DISTINCT messageY) AS yCount
RETURN 
  friendId, 
  friendFirstName, 
  friendLastName, 
  xCount, 
  yCount, 
  xCount + yCount AS xyCount
ORDER BY xyCount DESC, friendId ASC
LIMIT {6}
\end{verbatim}
}

\subsection{Query 4}

{\footnotesize
\begin{verbatim}
MATCH (person:Person {id:{1}})-[:KNOWS]-(:Person)<-[:HAS_CREATOR]-(post:Post)-[HAS_TAG]->(tag:Tag)
WHERE post.creationDate >= {2} AND post.creationDate < {3}
OPTIONAL MATCH (tag)<-[:HAS_TAG]-(oldPost:Post)
WHERE oldPost.creationDate < {2}
WITH tag, post, length(collect(oldPost)) AS oldPostCount
WHERE oldPostCount=0
RETURN 
  tag.name AS tagName, 
  length(collect(post)) AS postCount
ORDER BY postCount DESC, tagName ASC
LIMIT {4}
\end{verbatim}
}

\subsection{Query 5}

{\footnotesize
\begin{verbatim}
MATCH (person:Person {id:{1}})-[:KNOWS*1..2]-(friend:Person)<-[membership:HAS_MEMBER]-(forum:Forum)
WHERE membership.joinDate>{2} AND not(person=friend)
WITH DISTINCT friend, forum
OPTIONAL MATCH (friend)<-[:HAS_CREATOR]-(post:Post)<-[:CONTAINER_OF]-(forum)
WITH forum, count(post) AS postCount
RETURN 
  forum.title AS forumName, 
  postCount
ORDER BY postCount DESC, forum.id ASC
LIMIT {3}
\end{verbatim}
}

\subsection{Query 6}

{\footnotesize
\begin{verbatim}
MATCH (person:Person {id:{1}})-[:KNOWS*1..2]-(friend:Person),
      (friend)<-[:HAS_CREATOR]-(friendPost:Post)-[:HAS_TAG]->(knownTag:Tag {name:{2}})
WHERE not(person=friend)
MATCH (friendPost)-[:HAS_TAG]->(commonTag:Tag)
WHERE not(commonTag=knownTag)
WITH DISTINCT commonTag, knownTag, friend
MATCH (commonTag)<-[:HAS_TAG]-(commonPost:Post)-[:HAS_TAG]->(knownTag)
WHERE (commonPost)-[:HAS_CREATOR]->(friend)
RETURN 
  commonTag.name AS tagName, 
  count(commonPost) AS postCount
ORDER BY postCount DESC, tagName ASC
LIMIT {3}
\end{verbatim}
}

\subsection{Query 7}

{\footnotesize
\begin{verbatim}
MATCH (person:Person {id:{1}})<-[:HAS_CREATOR]-(message)<-[like:LIKES]-(liker:Person)
WITH liker, message, like.creationDate AS likeTime, person
ORDER BY likeTime DESC, message.id ASC
WITH liker, head(collect({msg: message, likeTime: likeTime})) AS latestLike, person
RETURN 
  liker.id AS personId, 
  liker.firstName AS personFirstName, 
  liker.lastName AS personLastName, 
  latestLike.likeTime AS likeTime, 
  not((liker)-[:KNOWS]-(person)) AS isNew, 
  latestLike.msg.id AS messageId, 
  latestLike.msg.content AS messageContent, 
  latestLike.likeTime - latestLike.msg.creationDate AS latencyAsMilli
ORDER BY likeTime DESC, personId ASC
LIMIT {2}
\end{verbatim}
}

\subsection{Query 8}

{\footnotesize
\begin{verbatim}
MATCH (start:Person {id:{1}})<-[:HAS_CREATOR]-()<-[:REPLY_OF]-(comment:Comment)-[:HAS_CREATOR]->(person:Person)
RETURN 
  person.id AS personId, 
  person.firstName AS personFirstName, 
  person.lastName AS personLastName, 
  comment.id AS commentId, 
  comment.creationDate AS commentCreationDate, 
  comment.content AS commentContent
ORDER BY commentCreationDate DESC, commentId ASC
LIMIT {2}
\end{verbatim}
}

\subsection{Query 9}

{\footnotesize
\begin{verbatim}
MATCH (:Person {id:{1}})-[:KNOWS*1..2]-(friend:Person)<-[:HAS_CREATOR]-(message)
WHERE message.creationDate < {2}
RETURN DISTINCT 
  message.id AS messageId, 
  CASE has(message.content) 
    WHEN true THEN message.content 
    ELSE message.imageFile 
  END AS messageContent,
  message.creationDate AS messageCreationDate, 
  friend.id AS personId, 
  friend.firstName AS personFirstName, 
  friend.lastName AS personLastName
ORDER BY message.creationDate DESC, message.id ASC
LIMIT {3}
\end{verbatim}
}

\subsection{Query 10}

{\footnotesize
\begin{verbatim}
MATCH (person:Person {id:{1}})-[:KNOWS*2..2]-(friend:Person)-[:IS_LOCATED_IN]->(city:City)
WHERE ((friend.birthday_month = {2} AND friend.birthday_day >= 21) 
      OR (friend.birthday_month = ({2}+1)%12 AND friend.birthday_day < 22)) 
      AND not(friend=person) 
      AND not((friend)-[:KNOWS]-(person))
WITH DISTINCT friend, city, person
OPTIONAL MATCH (friend)<-[:HAS_CREATOR]-(post:Post)
WITH friend, city, collect(post) AS posts, person
WITH 
  friend, 
  city, 
  length(posts) AS postCount, 
  length([p IN posts WHERE (p)-[:HAS_TAG]->(:Tag)<-[:HAS_INTEREST]-(person)]) AS commonPostCount
RETURN 
  friend.id AS personId, 
  friend.firstName AS personFirstName, 
  friend.lastName AS personLastName, 
  friend.gender AS personGender, 
  city.name AS personCityName, 
  commonPostCount - (postCount - commonPostCount) AS commonInterestScore
ORDER BY commonInterestScore DESC, personId ASC
LIMIT {4}
\end{verbatim}
}

\subsection{Query 11}

{\footnotesize
\begin{verbatim}
MATCH (person:Person {id:{1}})-[:KNOWS*1..2]-(friend:Person)
WHERE not(person=friend)
WITH DISTINCT friend
MATCH (friend)-[worksAt:WORKS_AT]->(company:Company)-[:IS_LOCATED_IN]->(:Country {name:{3}})
WHERE worksAt.workFrom < {2}
RETURN 
  friend.id AS friendId, 
  friend.firstName AS friendFirstName, 
  friend.lastName AS friendLastName, 
  worksAt.workFrom AS workFromYear, 
  company.name AS companyName
ORDER BY workFromYear ASC, friendId ASC, companyName DESC
LIMIT {4}
\end{verbatim}
}

\subsection{Query 12}

{\footnotesize
\begin{verbatim}
MATCH (:Person {id:{1}})-[:KNOWS]-(friend:Person)
OPTIONAL MATCH (friend)<-[:HAS_CREATOR]-(comment:Comment)-[:REPLY_OF]->(:Post)-[:HAS_TAG]->(tag:Tag),
               (tag)-[:HAS_TYPE]->(tagClass:TagClass)-[:IS_SUBCLASS_OF*0..]->(baseTagClass:TagClass)
WHERE tagClass.name = {2} OR baseTagClass.name = {2}
RETURN 
  friend.id AS friendId, 
  friend.firstName AS friendFirstName, 
  friend.lastName AS friendLastName, 
  collect(DISTINCT tag.name) AS tagNames, 
  count(DISTINCT comment) AS count
ORDER BY count DESC, friendId ASC
LIMIT {3}
\end{verbatim}
}

\subsection{Query 13}

{\footnotesize
\begin{verbatim}
MATCH (person1:Person {id:{1}}), (person2:Person {id:{2}})
OPTIONAL MATCH path = shortestPath((person1)-[:KNOWS]-(person2))
RETURN CASE path IS NULL
         WHEN true THEN -1 
         ELSE length(path) 
       END AS pathLength
\end{verbatim}
}

\subsection{Query 14}

{\footnotesize
\begin{verbatim}
MATCH path = allShortestPaths((person1:Person {id:{1}})-[:KNOWS]-(person2:Person {id:{2}}))
WITH nodes(path) AS pathNodes
RETURN
 extract(n IN pathNodes | n.id) AS pathNodeIds,
 reduce(weight=0.0, idx IN range(1,size(pathNodes)-1) |
    extract(prev IN [pathNodes[idx-1]] |
        extract(curr IN [pathNodes[idx]] |
            weight +
            length((curr)<-[:HAS_CREATOR]-(:Comment)-[:REPLY_OF]->(:Post)-[:HAS_CREATOR]->(prev))*1.0 +
            length((prev)<-[:HAS_CREATOR]-(:Comment)-[:REPLY_OF]->(:Post)-[:HAS_CREATOR]->(curr))*1.0 +
            length((prev)-[:HAS_CREATOR]-(:Comment)-[:REPLY_OF]-(:Comment)-[:HAS_CREATOR]-(curr))*0.5
        )
    )[0][0]
 ) AS weight
ORDER BY weight DESC
\end{verbatim}
}

\section{Sparksee API}

\subsection{Query 1}
{\footnotesize
\begin{verbatim}
List<LdbcQuery1Result> result = new ArrayList<LdbcQuery1Result>();
Graph graph = sess.getGraph();
Value v = new Value();

int personType       = SparkseeUtils.getType(SparkseeUtils.PERSON);
int organisationType = SparkseeUtils.getType(SparkseeUtils.ORGANISATION);
int isLocatedInType  = SparkseeUtils.getType(SparkseeUtils.IS_LOCATED_IN);
int workAtType       = SparkseeUtils.getType(SparkseeUtils.WORK_AT);
int studyAtType      = SparkseeUtils.getType(SparkseeUtils.STUDY_AT);
int speaksType       = SparkseeUtils.getType(SparkseeUtils.SPEAKS);
int emailType        = SparkseeUtils.getType(SparkseeUtils.EMAIL);
int placeType        = SparkseeUtils.getType(SparkseeUtils.PLACE);
int emailaddressType = SparkseeUtils.getType(SparkseeUtils.EMAILADDRESS_TYPE);
int languageType     = SparkseeUtils.getType(SparkseeUtils.LANGUAGE_TYPE);
int knowsType        = SparkseeUtils.getType(SparkseeUtils.KNOWS);

int workFromAttr     = SparkseeUtils.getAttribute(workAtType,  SparkseeUtils.WORK_FROM);
int classYearAttr    = SparkseeUtils.getAttribute(studyAtType, SparkseeUtils.CLASS_YEAR);
int placeNameAttr    = SparkseeUtils.getAttribute(placeType,  SparkseeUtils.NAME);
int personIdAttr     = SparkseeUtils.getAttribute(personType, SparkseeUtils.ID);
int firstNameAttr    = SparkseeUtils.getAttribute(personType, SparkseeUtils.FIRST_NAME);
int lastNameAttr     = SparkseeUtils.getAttribute(personType, SparkseeUtils.LAST_NAME);
int birthdayAttr     = SparkseeUtils.getAttribute(personType, SparkseeUtils.BIRTHDAY);
int locationIPAttr   = SparkseeUtils.getAttribute(personType, SparkseeUtils.LOCATION_IP);
int browserUsedAttr  = SparkseeUtils.getAttribute(personType, SparkseeUtils.BROWSER_USED);
int creationDateAttr = SparkseeUtils.getAttribute(personType, SparkseeUtils.CREATION_DATE);
int genderAttr       = SparkseeUtils.getAttribute(personType, SparkseeUtils.GENDER);
int emailAttr        = SparkseeUtils.getAttribute(emailaddressType, 
                                                  SparkseeUtils.EMAILADDRESS_ATTR);
int languageAttr     = SparkseeUtils.getAttribute(languageType,
                                                  SparkseeUtils.LANGUAGE_ATTR);
int organisationNameAttr = SparkseeUtils.getAttribute(organisationType, 
                                                      SparkseeUtils.NAME);
int distanceAttr = graph.newSessionAttribute(personType, DataType.Integer, 
                                             AttributeKind.Indexed);

v.setStringVoid(firstName); 
Value nullValue = new Value(); 
Value distanceValue = new Value(); 
for (int i = 0; i < 3 && result.size() < limit; i++) { 
distanceValue.setInteger(i+1); 
Objects tmp = graph.neighbors(knownPeople, knowsType, EdgesDirection.Outgoing); 
knownPeople.close(); 
knownPeople = tmp; 
Objects knownPeopleSharedName = graph.select(firstNameAttr, Condition.Equal, v, knownPeople); 
knownPeopleSharedName.remove(personOID); 
Objects knownPeopleSharedNameNotVisitedYet = graph.select(distanceAttr, 
                                                          Condition.Equal, 
                                                          nullValue, 
                                                          knownPeopleSharedName); 
knownPeopleSharedName.close(); 

int distanceFromPerson = i + 1; 
SimpleDateFormat dateFormat = new SimpleDateFormat(SparkseeUtils.DATE_FORMAT); 
ObjectsIterator iterator = knownPeopleSharedNameNotVisitedYet.iterator(); 
while (iterator.hasNext()) { 
    long oid = iterator.next(); 
    graph.setAttribute(oid, distanceAttr, distanceValue); 

    graph.getAttribute(oid, personIdAttr, v); 
    long friendId = v.getLong();  
    graph.getAttribute(oid, lastNameAttr, v); 
    String friendLastName = v.getString(); 

    graph.getAttribute(oid, birthdayAttr, v); 
    Date date = new Date(); 
    try { 
        date = dateFormat.parse(v.getString()); 
    } catch (ParseException ex) { 
    } 
    long friendBirthday = date.getTime(); 

    graph.getAttribute(oid, creationDateAttr, v); 
    long friendCreationDate = v.getTimestamp(); 

    graph.getAttribute(oid, genderAttr, v); 
    String friendGender = v.getString(); 

    graph.getAttribute(oid, browserUsedAttr, v); 
    String friendBrowserUsed = v.getString(); 

    graph.getAttribute(oid, locationIPAttr, v); 
    String friendLocationIp = v.getString(); 

    List<String> friendEmails = new ArrayList<String>(); 
    Objects emails = graph.neighbors(oid, emailType, EdgesDirection.Outgoing); 
    ObjectsIterator emailIt = emails.iterator(); 
    while (emailIt.hasNext()) {
        long emailOid = emailIt.next();
        graph.getAttribute(emailOid, emailAttr, v);
        friendEmails.add(v.getString());
    }
    emailIt.close();
    emails.close();

    List<String> friendLanguages = new ArrayList<String>();
    Objects languages = graph.neighbors(oid, speaksType, EdgesDirection.Outgoing);
    ObjectsIterator languagesIt = languages.iterator();
    while (languagesIt.hasNext()) {
        long languageOid = languagesIt.next();
        graph.getAttribute(languageOid, languageAttr, v);
        friendLanguages.add(v.getString());
    }
    languagesIt.close();
    languages.close();

    Objects locations = graph.neighbors(oid, isLocatedInType, EdgesDirection.Outgoing);
    long friendCity = locations.any();
    locations.close();
    graph.getAttribute(friendCity, placeNameAttr, v);
    String friendCityName = v.getString();

    List<List<Object>> friendUniversities = new ArrayList<List<Object>>();
    Objects studiedAt = graph.explode(oid, studyAtType, EdgesDirection.Outgoing);
    ObjectsIterator studiedAtIt = studiedAt.iterator();
    while (studiedAtIt.hasNext()) {
        long studiedAtOid = studiedAtIt.next();
        long universityOid = graph.getEdgeData(studiedAtOid).getHead();
        List<Object> studyAtData = new ArrayList<Object>();

        graph.getAttribute(universityOid, organisationNameAttr, v);
        studyAtData.add(v.getString());

        graph.getAttribute(studiedAtOid, classYearAttr, v);
        studyAtData.add(v.getInteger());

        locations = graph.neighbors(universityOid, 
                                    isLocatedInType, 
                                    EdgesDirection.Outgoing);
        long universityPlace = locations.any();
        locations.close();
        locations.close();
        graph.getAttribute(universityPlace, placeNameAttr, v);
        studyAtData.add(v.getString());
        friendUniversities.add(studyAtData);
    }
    studiedAtIt.close();
    studiedAt.close();

    List<List<Object>> friendCompanies = new ArrayList<List<Object>>();
    Objects workedAt = graph.explode(oid, workAtType, EdgesDirection.Outgoing);
    ObjectsIterator workedAtIt = workedAt.iterator();
    while (workedAtIt.hasNext()) {
        long workedAtOid = workedAtIt.next();
        long companyOid = graph.getEdgeData(workedAtOid).getHead();
        List<Object> workAtData = new ArrayList<Object>();

        graph.getAttribute(companyOid, organisationNameAttr, v);
        workAtData.add(v.getString());

        graph.getAttribute(workedAtOid, workFromAttr, v);
        workAtData.add(v.getInteger());

        locations = graph.neighbors(companyOid, isLocatedInType, EdgesDirection.Outgoing);
        long  companyPlace = locations.any();
        locations.close();
        graph.getAttribute(companyPlace, placeNameAttr, v);
        workAtData.add(v.getString());
        friendCompanies.add(workAtData);
    }
    workedAtIt.close();
    workedAt.close();

    result.add(new LdbcQuery1Result(
    friendId,
    friendLastName,
    distanceFromPerson,
    friendBirthday,
    friendCreationDate,
    friendGender,
    friendBrowserUsed,
    friendLocationIp,
    friendEmails,
    friendLanguages,
    friendCityName,
    friendUniversities,
    friendCompanies));
}
iterator.close();
knownPeopleSharedNameNotVisitedYet.close();
}
knownPeople.close();

Collections.sort(result, new Comparator<LdbcQuery1Result>() {
    public int compare(LdbcQuery1Result r1, LdbcQuery1Result r2) {
        Integer distance = r1.distanceFromPerson();
        // ascending by their distance from the start Person
        int rc = distance.compareTo(r2.distanceFromPerson()); 
        if (rc == 0) {
            // sort ascending by their last name
            rc = r1.friendLastName().compareTo(r2.friendLastName()); 
            if (rc == 0) {
                Long id = r1.friendId();
                // ascending by their identifier
                rc = id.compareTo(r2.friendId()); 
            }
        }
        return rc;
    }
});

sess.close();
return result.subList(0, Math.min(limit, result.size()));
}

\end{verbatim}
}

\subsection{Query 2}
{\footnotesize
\begin{verbatim}
List<LdbcQuery2Result> result = new ArrayList<LdbcQuery2Result>();
Graph graph = sess.getGraph();
Value v = new Value();

int personType     = SparkseeUtils.getType(SparkseeUtils.PERSON);
int postType       = SparkseeUtils.getType(SparkseeUtils.POST);
int commentType    = SparkseeUtils.getType(SparkseeUtils.COMMENT);
int hasCreatorType = SparkseeUtils.getType(SparkseeUtils.HAS_CREATOR);

int personIdAttr   = SparkseeUtils.getAttribute(personType, SparkseeUtils.ID);
int firstNameAttr  = SparkseeUtils.getAttribute(personType, SparkseeUtils.FIRST_NAME);
int lastNameAttr   = SparkseeUtils.getAttribute(personType, SparkseeUtils.LAST_NAME);
int postIdAttr     = SparkseeUtils.getAttribute(postType, SparkseeUtils.ID);
int creationDateAttr = SparkseeUtils.getAttribute(postType, SparkseeUtils.CREATION_DATE);
int postContentAttr  = SparkseeUtils.getAttribute(postType, SparkseeUtils.CONTENT);
int postImageFileAttr = SparkseeUtils.getAttribute(postType, SparkseeUtils.IMAGE_FILE);
int commentIdAttr     = SparkseeUtils.getAttribute(commentType, SparkseeUtils.ID);
int commentCreationDateAttr = SparkseeUtils.getAttribute(commentType, 
                                                         SparkseeUtils.CREATION_DATE);
int commentContentAttr = SparkseeUtils.getAttribute(commentType, 
                                                    SparkseeUtils.CONTENT);


Objects friends = SparkseeUtils.getKnownPeople(graph, v, personId, 1);

v.setTimestamp(maxDate.getTime());
Objects allPosts = graph.neighbors(friends, hasCreatorType, EdgesDirection.Ingoing);
Objects posts    = graph.select(creationDateAttr, Condition.LessEqual, v, allPosts);
Objects comments = graph.select(commentCreationDateAttr, Condition.LessEqual, v, allPosts);
allPosts.close();
friends.close();

ObjectsIterator iterator = posts.iterator();
while (iterator.hasNext()) {
    long oid = iterator.next();

    graph.getAttribute(oid, postIdAttr, v);
    long postId = v.getLong();

    graph.getAttribute(oid, creationDateAttr, v);
    long creationDate = v.getTimestamp();

    graph.getAttribute(oid, postContentAttr, v);
    if (v.isNull()) {
        graph.getAttribute(oid, postImageFileAttr, v);
    }
    String content = v.getString();

    Objects creator = graph.neighbors(oid, hasCreatorType, EdgesDirection.Outgoing);
    long creatorOid = creator.any();
    creator.close();
    graph.getAttribute(creatorOid, personIdAttr, v);
    long creatorId = v.getLong();
    graph.getAttribute(creatorOid, firstNameAttr, v);
    String firstName = v.toString();
    graph.getAttribute(creatorOid, lastNameAttr, v);
    String lastName = v.toString();

    result.add(new LdbcQuery2Result(creatorId, firstName, lastName,
    postId, content, creationDate));
}
iterator.close();
posts.close();

iterator = comments.iterator();
while (iterator.hasNext()) {
    long oid = iterator.next();

    graph.getAttribute(oid, commentIdAttr, v);
    long commentId = v.getLong();

    graph.getAttribute(oid, commentCreationDateAttr, v);
    long creationDate = v.getTimestamp();

    graph.getAttribute(oid, commentContentAttr, v);
    String content = v.getString();

    Objects creator = graph.neighbors(oid, hasCreatorType, EdgesDirection.Outgoing);
    long creatorOid = creator.any();
    creator.close();
    graph.getAttribute(creatorOid, personIdAttr, v);
    long creatorId = v.getLong();
    graph.getAttribute(creatorOid, firstNameAttr, v);
    String firstName = v.toString();
    graph.getAttribute(creatorOid, lastNameAttr, v);
    String lastName = v.toString();

    result.add(new LdbcQuery2Result(creatorId, firstName, lastName,
    commentId, content, creationDate));
}
iterator.close();
comments.close();
Collections.sort(result, new Comparator<LdbcQuery2Result>() {
    public int compare(LdbcQuery2Result r1, LdbcQuery2Result r2) {
        Long date = r2.postOrCommentCreationDate();
        // descending by creation date
        int rc = date.compareTo(r1.postOrCommentCreationDate()); 
        if (rc == 0) {
            Long id = r1.postOrCommentId();
            // ascending by Post identifier
            rc = id.compareTo(r2.postOrCommentId()); 
        }
        return rc;
    }
});

sess.close();
return result.subList(0, Math.min(limit, result.size()));

\end{verbatim}
}

\subsection{Query 3}

{\footnotesize
\begin{verbatim}
List<LdbcQuery3Result> result = new ArrayList<LdbcQuery3Result>();
Graph graph = sess.getGraph();
Value v = new Value();

Calendar cal = Calendar.getInstance();
cal.setTime(startDate);
cal.add(Calendar.DATE, durationDays);
Date endDate = cal.getTime();

int personType      = SparkseeUtils.getType(SparkseeUtils.PERSON);
int placeType       = SparkseeUtils.getType(SparkseeUtils.PLACE);
int postType        = SparkseeUtils.getType(SparkseeUtils.POST);
int commentType     = SparkseeUtils.getType(SparkseeUtils.COMMENT);
int isLocatedInType = SparkseeUtils.getType(SparkseeUtils.IS_LOCATED_IN);
int hasCreatorType  = SparkseeUtils.getType(SparkseeUtils.HAS_CREATOR);
int isPartOfType    = SparkseeUtils.getType(SparkseeUtils.IS_PART_OF);

int personIdAttr         = SparkseeUtils.getAttribute(personType, SparkseeUtils.ID);
int firstNameAttr        = SparkseeUtils.getAttribute(personType, SparkseeUtils.FIRST_NAME);
int lastNameAttr         = SparkseeUtils.getAttribute(personType, SparkseeUtils.LAST_NAME);
int countryNameAttr      = SparkseeUtils.getAttribute(placeType,  SparkseeUtils.NAME);
int postCreationDateAttr = SparkseeUtils.getAttribute(postType, 
                                                         SparkseeUtils.CREATION_DATE);
int commentCreationDateAttr = SparkseeUtils.getAttribute(commentType,
                                                         SparkseeUtils.CREATION_DATE);

Objects friends = SparkseeUtils.getKnownPeople(graph, v, personId, 2);
long country1OID = graph.findObject(countryNameAttr, v.setString(countryXName));
long country2OID = graph.findObject(countryNameAttr, v.setString(countryYName));

Objects postsCountry1 = graph.neighbors(country1OID, isLocatedInType, 
                                        EdgesDirection.Ingoing);
Objects postsCountry2 = graph.neighbors(country2OID, isLocatedInType, 
                                        EdgesDirection.Ingoing);

Value vFrom = new Value();
vFrom.setTimestampVoid(startDate.getTime());
Value vTo = new Value();
vTo.setTimestampVoid(endDate.getTime());

//search matches
ObjectsIterator iterator = friends.iterator();
while (iterator.hasNext()) {
    long oid = iterator.next();
    Objects city = graph.neighbors(oid, isLocatedInType, EdgesDirection.Outgoing);
    long cityOid = city.any();
    city.close();
    Objects countries = graph.neighbors(cityOid, isPartOfType, EdgesDirection.Outgoing);
    long countryOid = countries.any();
    countries.close();
    if (countryOid != country1OID && countryOid != country2OID) {
        Objects posts = graph.neighbors(oid, hasCreatorType, 
                                        EdgesDirection.Ingoing);
        Objects filter  = graph.select(postCreationDateAttr, 
                                       Condition.Between, vFrom, vTo, posts);
        Objects filter2 = graph.select(commentCreationDateAttr, 
                                       Condition.Between, vFrom, vTo, posts);
        filter.union(filter2);
        Objects fromFirst  = Objects.combineIntersection(filter, 
                                                         postsCountry1);
        Objects fromSecond = Objects.combineIntersection(filter, 
                                                         postsCountry2);
        filter.close();
        filter2.close();

        if (fromFirst.size() > 0 && fromSecond.size() > 0) {
            graph.getAttribute(oid, personIdAttr, v);
            long friendId = v.getLong();
            graph.getAttribute(oid, firstNameAttr, v);
            String personFirstName = v.getString();
            graph.getAttribute(oid, lastNameAttr, v);
            String personLastName = v.getString();

            result.add(new LdbcQuery3Result(
            friendId, personFirstName, personLastName,
            fromFirst.size(), fromSecond.size(),
            fromFirst.size() + fromSecond.size()));
        }

        posts.close();
        fromFirst.close();
        fromSecond.close();
    }
    city.close();
    countries.close();
}
iterator.close();
friends.close();
postsCountry1.close();
postsCountry2.close();

Collections.sort(result, new Comparator<LdbcQuery3Result>() {
    public int compare(LdbcQuery3Result r1, LdbcQuery3Result r2) {
        Long count = r2.count();
        // descending by total number of Posts/Comments
        int rc = count.compareTo(r1.count());         
        if (rc == 0) {
            Long id = r1.personId();
            // ascending by Person identifier.
            rc = id.compareTo(r2.personId());        
        }
        return rc;
    }
});
sess.close();
return result.subList(0, Math.min(limit, result.size()));

\end{verbatim}
}

\subsection{Query 4}

{\footnotesize
\begin{verbatim}
List<LdbcQuery4Result> result = new ArrayList<LdbcQuery4Result>();
Graph graph = sess.getGraph();
Value v = new Value();

Calendar cal = Calendar.getInstance();
cal.setTime(startDate);
cal.add(Calendar.DATE, durationDays - 1);
Date endDate = cal.getTime();

Value vFrom = new Value();
vFrom.setTimestampVoid(startDate.getTime());
Value vTo = new Value();
vTo.setTimestampVoid(endDate.getTime());

int postType       = SparkseeUtils.getType(SparkseeUtils.POST);
int hasCreatorType = SparkseeUtils.getType(SparkseeUtils.HAS_CREATOR);
int hasTagType     = SparkseeUtils.getType(SparkseeUtils.HAS_TAG);
int tagType        = SparkseeUtils.getType(SparkseeUtils.TAG);

int creationDateAttr = SparkseeUtils.getAttribute(postType,   SparkseeUtils.CREATION_DATE);
int tagNameAttr      = SparkseeUtils.getAttribute(tagType,    SparkseeUtils.NAME);

Objects friends = SparkseeUtils.getKnownPeople(graph, v, personId, 1);

Objects postsPreDate  = graph.select(creationDateAttr, Condition.LessThan, vFrom);
Objects postsPostDate = graph.select(creationDateAttr, Condition.Between,  vFrom, vTo);

Objects posts = graph.neighbors(friends, hasCreatorType, EdgesDirection.Ingoing);
Objects allPosts = graph.select(postType);
posts.intersection(allPosts);
allPosts.close();
friends.close();

Objects postPosts = Objects.combineIntersection(posts, postsPostDate);
posts.close();

Objects preTags = graph.neighbors(postsPreDate, hasTagType, EdgesDirection.Outgoing);

Objects postTags = graph.neighbors(postPosts, hasTagType, EdgesDirection.Outgoing);
postTags.difference(preTags);
preTags.close();

postsPreDate.close();
postsPostDate.close();
ObjectsIterator iterator = postTags.iterator();
while (iterator.hasNext()) {
    long oid = iterator.next();

    graph.getAttribute(oid, tagNameAttr, v);
    String tagName = v.getString();
    Objects friendPosts = graph.neighbors(oid, hasTagType, EdgesDirection.Ingoing);
    friendPosts.intersection(postPosts);

    result.add(new LdbcQuery4Result(tagName, (int) friendPosts.count()));
    friendPosts.close();
}
iterator.close();
postPosts.close();
postTags.close();

Collections.sort(result, new Comparator<LdbcQuery4Result>() {
    public int compare(LdbcQuery4Result r1, LdbcQuery4Result r2) {
        Integer count = r2.postCount();
        int rc = count.compareTo(r1.postCount()); // Sort results descending by Post count
        if (rc == 0) {
            rc = r1.tagName().compareTo(r2.tagName()); //and then ascending by Tag name.
        }
        return rc;
    }
});

sess.close();
return result.subList(0, Math.min(limit, result.size()));

\end{verbatim}
}

\subsection{Query 5}

{\footnotesize
\begin{verbatim}
List<LdbcQuery5Result> result = new ArrayList<LdbcQuery5Result>();
Graph graph = sess.getGraph();
Value v = new Value();

int postType         = SparkseeUtils.getType(SparkseeUtils.POST);
int hasMemberType    = SparkseeUtils.getType(SparkseeUtils.HAS_MEMBER);
int containerOfType  = SparkseeUtils.getType(SparkseeUtils.CONTAINER_OF);
int forumType        = SparkseeUtils.getType(SparkseeUtils.FORUM);
int hasCreatorType   = SparkseeUtils.getType(SparkseeUtils.HAS_CREATOR);

int joinDateAttr   = SparkseeUtils.getAttribute(hasMemberType, SparkseeUtils.JOIN_DATE);
int forumIdAttr    = SparkseeUtils.getAttribute(forumType, SparkseeUtils.ID);
int forumTitleAttr = SparkseeUtils.getAttribute(forumType, SparkseeUtils.TITLE);

Objects friends = SparkseeUtils.getKnownPeople(graph, v, personId, 2);
Objects members = graph.explode(friends, hasMemberType, EdgesDirection.Ingoing);
v.setTimestampVoid(minDate.getTime());
Objects candidate = graph.select(joinDateAttr, Condition.GreaterThan, v, members);
members.close();
Objects posts = graph.neighbors(friends, hasCreatorType, EdgesDirection.Ingoing);
friends.close();
Objects allPosts = graph.select(postType);  // Only post
posts.intersection(allPosts);
allPosts.close();

Map<Long, Container> forums    = new HashMap<Long, Container>(candidate.size());
Map<Long, Objects> forumPost   = new HashMap<Long, Objects>(candidate.size());
Map<Long, Objects> friendPosts = new HashMap<Long, Objects>(candidate.size());
ObjectsIterator iterator = candidate.iterator();
while (iterator.hasNext()) {
    long oid = iterator.next();
    EdgeData edata = graph.getEdgeData(oid);

    if (!forumPost.containsKey(edata.getTail())) {
        forumPost.put(edata.getTail(), 
        graph.neighbors(edata.getTail(), 
        containerOfType,  
        EdgesDirection.Outgoing));
    }
    if (!friendPosts.containsKey(edata.getHead())) {
        friendPosts.put(edata.getHead(), 
        graph.neighbors(edata.getHead(),
        hasCreatorType, 
        EdgesDirection.Ingoing));
    }

    Objects memberPost = friendPosts.get(edata.getHead());
    Objects postsGroup = forumPost.get(edata.getTail());
    Objects friendForumPosts = Objects.combineIntersection(memberPost, postsGroup);
    friendForumPosts.intersection(posts);
    Objects friendForumPosts = Objects.combineIntersection(memberPost, postsGroup);
    friendForumPosts.intersection(posts);
    int count = friendForumPosts.size();
    friendForumPosts.close();
    Container container;
    if (forums.containsKey(edata.getTail())) {
        container = forums.get(edata.getTail());
    } else {
        graph.getAttribute(edata.getTail(), forumIdAttr, v);
        container = new Container(edata.getTail(), v.getLong(), 0);
    }
    container.add(count);
    forums.put(edata.getTail(), container);
}
iterator.close();
candidate.close();
posts.close();

for (Objects objs : forumPost.values()) {
    objs.close();
}

for (Objects objs : friendPosts.values()) {
    objs.close();
}

List<Container> toSort = new ArrayList<Container>(forums.values());
Collections.sort(toSort, new Comparator<Container>() {
    public int compare(Container r1, Container r2) {
        Integer count = r2.count();
        int rc = count.compareTo(r1.count()); // descending by the count of Posts
        if (rc == 0) {
            rc = r1.id.compareTo(r2.id()); // ascending by Forum title.
        }
        return rc;
    }
});

for (int i = 0; i < limit && i < toSort.size(); i++) {
    graph.getAttribute(toSort.get(i).oid(), forumTitleAttr,v);
    result.add(new LdbcQuery5Result(v.getString(), toSort.get(i).count()));
}
sess.close();
return result;
\end{verbatim}
}

\subsection{Query 6}

{\footnotesize
\begin{verbatim}
List<LdbcQuery6Result> result = new ArrayList<LdbcQuery6Result>();
Graph graph = sess.getGraph();
Value v = new Value();

int tagType    = SparkseeUtils.getType(SparkseeUtils.TAG);
int postType   = SparkseeUtils.getType(SparkseeUtils.POST);
int hasTag     = SparkseeUtils.getType(SparkseeUtils.HAS_TAG);
int hasCreator = SparkseeUtils.getType(SparkseeUtils.HAS_CREATOR);

int tagNameAttr = SparkseeUtils.getAttribute(tagType, SparkseeUtils.NAME);

v.setStringVoid(tagName);
long tagOID = graph.findObject(tagNameAttr, v);

Objects friends = SparkseeUtils.getKnownPeople(graph, v, personId, 2);
Objects posts = graph.neighbors(friends, hasCreator, EdgesDirection.Ingoing);
Objects allPost = graph.select(postType); // it's restricted to post only
posts.intersection(allPost);
allPost.close();
friends.close();
Objects postTag = graph.neighbors(tagOID, hasTag, EdgesDirection.Ingoing);
posts.intersection(postTag);
postTag.close();

Objects candidate = graph.neighbors(posts, hasTag, EdgesDirection.Outgoing);
candidate.remove(tagOID);

//search matches
ObjectsIterator iterator = candidate.iterator();
while (iterator.hasNext()) {
    long oid = iterator.next();

    graph.getAttribute(oid, tagNameAttr, v);
    Objects tagPosts = graph.neighbors(oid, hasTag, EdgesDirection.Ingoing);
    tagPosts.intersection(posts);

    result.add(new LdbcQuery6Result(v.getString(), (int)tagPosts.count()));
    tagPosts.close();
}
iterator.close();
candidate.close();
posts.close();

Collections.sort(result, new Comparator<LdbcQuery6Result>() {
    public int compare(LdbcQuery6Result r1, LdbcQuery6Result r2) {
        Integer count = r2.postCount();
        int rc = count.compareTo(r1.postCount()) ; // descending by count
        if (rc == 0) {
            rc = r1.tagName().compareTo(r2.tagName()); // ascending by Tag name
        }
        return rc;
    }
});

sess.close();
return result.subList(0, Math.min(limit, result.size()));

\end{verbatim}
}

\subsection{Query 7}

{\footnotesize
\begin{verbatim}
Graph graph = sess.getGraph();
Value v = new Value();

int hasCreatorType = SparkseeUtils.getType(SparkseeUtils.HAS_CREATOR);
int likeType       = SparkseeUtils.getType(SparkseeUtils.LIKES);
int personType     = SparkseeUtils.getType(SparkseeUtils.PERSON);
int postType       = SparkseeUtils.getType(SparkseeUtils.POST);
int commentType    = SparkseeUtils.getType(SparkseeUtils.COMMENT);

int personIdAttr         = SparkseeUtils.getAttribute(personType, SparkseeUtils.ID);
int nameAttr             = SparkseeUtils.getAttribute(personType, SparkseeUtils.FIRST_NAME);
int lastNameAttr         = SparkseeUtils.getAttribute(personType, SparkseeUtils.LAST_NAME);
int likeCreationDateAttr = SparkseeUtils.getAttribute(likeType, SparkseeUtils.CREATION_DATE);
int postIdAttr           = SparkseeUtils.getAttribute(postType, SparkseeUtils.ID);
int postCreationDateAttr = SparkseeUtils.getAttribute(postType, SparkseeUtils.CREATION_DATE);
int contentAttr          = SparkseeUtils.getAttribute(postType, SparkseeUtils.CONTENT);
int imageFileAttr        = SparkseeUtils.getAttribute(postType, SparkseeUtils.IMAGE_FILE);
int commentIdAttr        = SparkseeUtils.getAttribute(commentType, SparkseeUtils.ID);
int commentContentAttr   = SparkseeUtils.getAttribute(commentType, SparkseeUtils.CONTENT);
int commentCreationDateAttr = SparkseeUtils.getAttribute(commentType, 
                                                         SparkseeUtils.CREATION_DATE);

Objects friends = SparkseeUtils.getKnownPeople(graph, v, personId, 1);

long personOID = graph.findObject(personIdAttr, v.setLong(personId));
Objects posts = graph.neighbors(personOID, hasCreatorType, EdgesDirection.Ingoing);
Objects likes = graph.explode(posts,       likeType,       EdgesDirection.Ingoing);
posts.close();

Map<Long, LdbcQuery7Result> likeMap = new HashMap<Long, LdbcQuery7Result>();
ObjectsIterator iterator = likes.iterator();
while (iterator.hasNext()) {
    long oid = iterator.next();
    EdgeData edgeData = graph.getEdgeData(oid);

    graph.getAttribute(edgeData.getTail(), personIdAttr, v);
    long friendId = v.getLong();

    graph.getAttribute(oid, likeCreationDateAttr, v);
    long likeCreationDate = v.getTimestamp();

    if (!likeMap.containsKey(friendId) || 
        likeMap.get(friendId).likeCreationDate() > likeCreationDate) {
        graph.getAttribute(edgeData.getTail(), nameAttr, v);
        String firstName = v.getString();
        graph.getAttribute(edgeData.getTail(), lastNameAttr, v);
        String lastName = v.getString();

        boolean isNew = !friends.exists(edgeData.getTail());
        boolean isPost = (graph.getObjectType(edgeData.getHead()) == postType);
        graph.getAttribute(edgeData.getHead(), (isPost) ? postIdAttr : commentIdAttr, v);
        long commentOrPostId = v.getLong();

        graph.getAttribute(edgeData.getHead(), 
                          (isPost) ? postCreationDateAttr : commentCreationDateAttr, 
                           v);
        long latency = likeCreationDate - v.getTimestamp();

        graph.getAttribute(edgeData.getHead(), 
                           (isPost) ? contentAttr : commentContentAttr,
                           v);

        if (isPost && v.isNull()) {
        graph.getAttribute(edgeData.getHead(), imageFileAttr, v);
    }
    String commentOrPostContent = v.getString();

    likeMap.put(friendId, new LdbcQuery7Result(friendId, firstName,
                lastName, likeCreationDate, commentOrPostId,
                commentOrPostContent, (int)(latency / (1000L*60L)), isNew));
    }
}
iterator.close();
likes.close();
friends.close();


List<LdbcQuery7Result> result = new ArrayList<LdbcQuery7Result>(likeMap.values());
Collections.sort(result, new Comparator<LdbcQuery7Result>() {
    public int compare(LdbcQuery7Result r1, LdbcQuery7Result r2) {
        Long date = r2.likeCreationDate();
        // descending by creation time of Like
        int rc = date.compareTo(r1.likeCreationDate()); 
        // ascending by Person identifier of liker
        if (rc == 0) {
            Long id = r1.personId();
            rc = id.compareTo(r2.personId());
        }
        return rc;
    }
});

sess.close();
return result.subList(0, Math.min(limit, result.size()));

\end{verbatim}
}

\subsection{Query 8}

{\footnotesize
\begin{verbatim}
List<LdbcQuery8Result> result = new ArrayList<LdbcQuery8Result>();
Graph graph = sess.getGraph();
Value v = new Value();

int hasCreatorType = SparkseeUtils.getType(SparkseeUtils.HAS_CREATOR);
int replyOfType    = SparkseeUtils.getType(SparkseeUtils.REPLY_OF);
int personType     = SparkseeUtils.getType(SparkseeUtils.PERSON);
int postType       = SparkseeUtils.getType(SparkseeUtils.POST);
int commentType    = SparkseeUtils.getType(SparkseeUtils.COMMENT);

int personIdAttr          = SparkseeUtils.getAttribute(personType, SparkseeUtils.ID);
int firstNameAttr         = SparkseeUtils.getAttribute(personType, SparkseeUtils.FIRST_NAME);
int lastNameAttr          = SparkseeUtils.getAttribute(personType, SparkseeUtils.LAST_NAME);
int postIdAttr            = SparkseeUtils.getAttribute(postType, SparkseeUtils.ID);
int postContentAttr       = SparkseeUtils.getAttribute(postType, SparkseeUtils.CONTENT);
int postCreationDateAttr  = SparkseeUtils.getAttribute(postType, SparkseeUtils.CREATION_DATE);
int commentIdAttr         = SparkseeUtils.getAttribute(commentType, 
                                                       SparkseeUtils.ID);
int commentContentAttr    = SparkseeUtils.getAttribute(commentType, 
                                                       SparkseeUtils.CONTENT);
int commentCreationDateAttr = SparkseeUtils.getAttribute(commentType, 
                                                         SparkseeUtils.CREATION_DATE);

long personOID  = graph.findObject(personIdAttr, v.setLong(personId));
Objects posts   = graph.neighbors(personOID, hasCreatorType, EdgesDirection.Ingoing);
Objects replies = graph.neighbors(posts,     replyOfType,    EdgesDirection.Ingoing);
posts.close();

ObjectsIterator iterator = replies.iterator();
while (iterator.hasNext()) {
    long oid = iterator.next();
    boolean isPost = graph.getObjectType(oid) == postType;

    graph.getAttribute(oid, isPost ? postIdAttr : commentIdAttr, v);
    long messageId = v.getLong();

    graph.getAttribute(oid, isPost ? postCreationDateAttr : commentCreationDateAttr, v);
    long messageCreationDate = v.getTimestamp();

    graph.getAttribute(oid, isPost ? postContentAttr : commentContentAttr, v);
    String messageContent = v.getString();

    Objects creator = graph.neighbors(oid, hasCreatorType, EdgesDirection.Outgoing);
    long creatorOid = creator.any();
    creator.close();
    graph.getAttribute(creatorOid, personIdAttr, v);
    long creatorId = v.getLong();
    graph.getAttribute(creatorOid, firstNameAttr, v);
    String firstName = v.getString();
    graph.getAttribute(creatorOid, lastNameAttr, v);
    String lastName = v.getString();

    result.add(new LdbcQuery8Result(creatorId, firstName,
    lastName, messageCreationDate, messageId, messageContent));
}
iterator.close();
replies.close();

Collections.sort(result, new Comparator<LdbcQuery8Result>() {
    public int compare(LdbcQuery8Result r1, LdbcQuery8Result r2) {
        Long date = r2.commentCreationDate();
        // descending by creation date of reply Comment
        int rc = date.compareTo(r1.commentCreationDate()); 
        if (rc == 0) {
            Long id = r1.commentId();
            //ascending by identifier of reply Comment
            rc = id.compareTo(r2.commentId()); 
        }
        return rc;
    }
});

sess.close();
return result.subList(0, Math.min(limit, result.size()));

\end{verbatim}
}

\subsection{Query 9}

{\footnotesize
\begin{verbatim}
List<LdbcQuery9Result> result = new ArrayList<LdbcQuery9Result>();
Graph graph = sess.getGraph();
Value v = new Value();

int personType = SparkseeUtils.getType(SparkseeUtils.PERSON);
int postType = SparkseeUtils.getType(SparkseeUtils.POST);
int commentType = SparkseeUtils.getType(SparkseeUtils.COMMENT);
int hasCreatorType = SparkseeUtils.getType(SparkseeUtils.HAS_CREATOR);

int personIdAttr = SparkseeUtils.getAttribute(personType, SparkseeUtils.ID);
int firstNameAttr = SparkseeUtils.getAttribute(personType, SparkseeUtils.FIRST_NAME);
int lastNameAttr = SparkseeUtils.getAttribute(personType, SparkseeUtils.LAST_NAME);
int postIdAttr = SparkseeUtils.getAttribute(postType, SparkseeUtils.ID);
int creationDateAttr = SparkseeUtils.getAttribute(postType, SparkseeUtils.CREATION_DATE);
int postContentAttr = SparkseeUtils.getAttribute(postType, SparkseeUtils.CONTENT);
int postImageFileAttr = SparkseeUtils.getAttribute(postType, SparkseeUtils.IMAGE_FILE);
int commentIdAttr = SparkseeUtils.getAttribute(commentType, SparkseeUtils.ID);
int commentCreationDateAttr = SparkseeUtils.getAttribute(commentType, 
                                                         SparkseeUtils.CREATION_DATE);
int commentContentAttr = SparkseeUtils.getAttribute(commentType, SparkseeUtils.CONTENT);

Objects friends = SparkseeUtils.getKnownPeople(graph, v, personId, 2);

Value upLimit = new Value();
upLimit.setTimestamp(maxDate.getTime()-1);
Value downLimit = new Value();
downLimit.setTimestamp(0);
Values values = graph.getValues(creationDateAttr);
ValuesIterator vit = values.iterator();
if (vit.hasNext()) {
    downLimit = vit.next();
}
vit.close();
values.close();
values = graph.getValues(commentCreationDateAttr);
vit = values.iterator();
if (vit.hasNext()) {
    v = vit.next();
    if (v.getTimestamp() < downLimit.getTimestamp()) {
        downLimit = vit.next();
    }
}
vit.close();
values.close();

int partitions = 50;
long interval = (upLimit.getTimestamp() - downLimit.getTimestamp()) / partitions;
long base = downLimit.getTimestamp();
for (int i = 0; i < partitions && result.size() < limit; i++) {
Objects allPosts = graph.neighbors(friends, hasCreatorType, EdgesDirection.Ingoing);
Objects posts    = graph.select(creationDateAttr,        Condition.Between,
downLimit.setTimestamp(base+interval*(partitions-i-1)), 
                       upLimit.setTimestamp(base+interval*(partitions-i)), allPosts);
Objects comments = graph.select(commentCreationDateAttr, Condition.Between,
downLimit.setTimestamp(base+interval*(partitions-i-1)), 
                       upLimit.setTimestamp(base+interval*(partitions-i)), allPosts);
allPosts.close();

//search matches
ObjectsIterator iterator = posts.iterator();
while (iterator.hasNext()) {
    long oid = iterator.next();

    graph.getAttribute(oid, postIdAttr, v);
    long postId = v.getLong();
    graph.getAttribute(oid, creationDateAttr, v);
    long creationDate = v.getTimestamp();

    graph.getAttribute(oid, postContentAttr, v);
    if (v.isNull()) {
        graph.getAttribute(oid, postImageFileAttr, v);
    }
    String content = v.getString();

    Objects creator = graph.neighbors(oid, 
                                      hasCreatorType, 
                                      EdgesDirection.Outgoing);
    long creatorOid = creator.any();
    creator.close();
    graph.getAttribute(creatorOid, personIdAttr, v);
    Long creatorId = v.getLong();
    graph.getAttribute(creatorOid, firstNameAttr, v);
    String firstName = v.getString();
    graph.getAttribute(creatorOid, lastNameAttr, v);
    String lastName = v.getString();

    result.add(new LdbcQuery9Result(creatorId, firstName, lastName,
    postId, content, creationDate));
}
iterator.close();
posts.close();

iterator = comments.iterator();
while (iterator.hasNext()) {
    long oid = iterator.next();

    graph.getAttribute(oid, commentIdAttr, v);
    long commentId = v.getLong();
    graph.getAttribute(oid, commentCreationDateAttr, v);
    long creationDate = v.getTimestamp();
    graph.getAttribute(oid, commentContentAttr, v);
    String content = v.getString();

    Objects creator = graph.neighbors(oid, hasCreatorType, 
                                           EdgesDirection.Outgoing);
    long creatorOid = creator.any();
    creator.close();
    graph.getAttribute(creatorOid, personIdAttr, v);
    Long creatorId = v.getLong();
    graph.getAttribute(creatorOid, firstNameAttr, v);
    String firstName = v.getString();
    graph.getAttribute(creatorOid, lastNameAttr, v);
    String lastName = v.getString();

    result.add(new LdbcQuery9Result(creatorId, firstName, lastName,
    commentId, content, creationDate));
}
iterator.close();
comments.close();
}

friends.close();
Collections.sort(result, new Comparator<LdbcQuery9Result>() {
    public int compare(LdbcQuery9Result r1, LdbcQuery9Result r2) {
        Long date = r2.commentOrPostCreationDate();
        // descending by creation date
        int rc = date.compareTo(r1.commentOrPostCreationDate()); 
        if (rc == 0) {
            Long id = r1.commentOrPostId();
            // ascending by Post identifier
            rc = id.compareTo(r2.commentOrPostId()); 
        }
        return rc;
    }
});

sess.close();
return result.subList(0, Math.min(limit, result.size()));

\end{verbatim}
}

\subsection{Query 10}

{\footnotesize
\begin{verbatim}
List<LdbcQuery10Result> result = new ArrayList<LdbcQuery10Result>();
Graph graph = sess.getGraph();
Value v = new Value();

int personType      = SparkseeUtils.getType(SparkseeUtils.PERSON);
int postType        = SparkseeUtils.getType(SparkseeUtils.POST);
int hasCreatorType  = SparkseeUtils.getType(SparkseeUtils.HAS_CREATOR);
int hasTagType      = SparkseeUtils.getType(SparkseeUtils.HAS_TAG);
int placeType       = SparkseeUtils.getType(SparkseeUtils.PLACE);
int hasInterestType = SparkseeUtils.getType(SparkseeUtils.HAS_INTEREST);
int isLocatedInType = SparkseeUtils.getType(SparkseeUtils.IS_LOCATED_IN);

int personIdAttr  = SparkseeUtils.getAttribute(personType, SparkseeUtils.ID);
int birthdayAttr  = SparkseeUtils.getAttribute(personType, SparkseeUtils.BIRTHDAY);
int firstNameAttr = SparkseeUtils.getAttribute(personType, SparkseeUtils.FIRST_NAME);
int lastNameAttr  = SparkseeUtils.getAttribute(personType, SparkseeUtils.LAST_NAME);
int genderAttr    = SparkseeUtils.getAttribute(personType, SparkseeUtils.GENDER);
int placeNameAttr = SparkseeUtils.getAttribute(placeType,  SparkseeUtils.NAME);


int secondMonth = month + 1;
if (secondMonth > 12) {
    secondMonth = 1;
}

Objects friends = SparkseeUtils.getLastHopKnownPeople(sess, graph, v, personId, 2);

v.setLongVoid(personId);
long personOID   = graph.findObject(personIdAttr, v);
Objects userTags = graph.neighbors(personOID, hasInterestType, EdgesDirection.Outgoing);
Objects posts    = graph.select(postType);

//search matches
Date date = new Date();
Calendar calendar = Calendar.getInstance();
SimpleDateFormat dateFormat = new SimpleDateFormat(SparkseeUtils.DATE_FORMAT);
ObjectsIterator iterator = friends.iterator();
while (iterator.hasNext()) {
    long oid = iterator.next();

    graph.getAttribute(oid, birthdayAttr, v);
    try {
        date = dateFormat.parse(v.toString());
    } catch (ParseException ex) {
    }
    calendar.setTime(date);
    int birthMonth = calendar.get(Calendar.MONTH) + 1;
    int day = calendar.get(Calendar.DAY_OF_MONTH);
    if ((birthMonth == month && day >= 21) || (birthMonth == (secondMonth) && day < 22)) {

        Objects friendPosts = graph.neighbors(oid, hasCreatorType, EdgesDirection.Ingoing);
        friendPosts.intersection(posts); // Posts only

        int score = 0;
        ObjectsIterator postIt = friendPosts.iterator();
        while (postIt.hasNext()) {
            long postOID = postIt.next();

            Objects postTag = graph.neighbors(postOID, 
                                              hasTagType, 
                                              EdgesDirection.Outgoing);
            boolean sharedInterest = false;
            ObjectsIterator tagIt = postTag.iterator();
            while (tagIt.hasNext() && !sharedInterest) {
                long tagOID = tagIt.next();
                if (userTags.exists(tagOID)) {
                    sharedInterest = true;
                }
            }
            score += (sharedInterest) ? 1 : -1;
            tagIt.close();
            postTag.close();
        }
        postIt.close();

        graph.getAttribute(oid, personIdAttr, v);
        long friendId = v.getLong();
        graph.getAttribute(oid, firstNameAttr, v);
        String firstName = v.getString();
        graph.getAttribute(oid, lastNameAttr, v);
        String lastName = v.getString();
        graph.getAttribute(oid, genderAttr, v);
        String gender = v.getString();
        Objects userPlace = graph.neighbors(oid, 
                                            isLocatedInType, 
                                            EdgesDirection.Outgoing);
        long placeOID = userPlace.any();
        userPlace.close();
        graph.getAttribute(placeOID, placeNameAttr, v);
        String placeName = v.getString();
        
        result.add(new LdbcQuery10Result(friendId, firstName, lastName,
        score, gender, placeName));
        friendPosts.close();
    }
}
iterator.close();
friends.close();
userTags.close();
posts.close();
Collections.sort(result, new Comparator<LdbcQuery10Result>() {
    public int compare(LdbcQuery10Result r1, LdbcQuery10Result r2) {
        Integer score = r2.commonInterestScore();
        // descending by similarity score
        int rc = score.compareTo(r1.commonInterestScore()) ; 
        if (rc == 0) {
            Long id = r1.personId();
            // ascending by Person identifier
            rc = id.compareTo(r2.personId()); 
        }
        return rc;
    }
});

sess.close();
return result.subList(0, Math.min(limit, result.size()));

\end{verbatim}
}

\subsection{Query 11}

{\footnotesize
\begin{verbatim}
List<LdbcQuery11Result> result = new ArrayList<LdbcQuery11Result>();
Graph graph = sess.getGraph();
Value v = new Value();

int personType       = SparkseeUtils.getType(SparkseeUtils.PERSON);
int locationType     = SparkseeUtils.getType(SparkseeUtils.PLACE);
int isLocatedInType  = SparkseeUtils.getType(SparkseeUtils.IS_LOCATED_IN);
int workAtType       = SparkseeUtils.getType(SparkseeUtils.WORK_AT);
int organisationType = SparkseeUtils.getType(SparkseeUtils.ORGANISATION);

int personIdAttr    = SparkseeUtils.getAttribute(personType, SparkseeUtils.ID);
int firstNameAttr   = SparkseeUtils.getAttribute(personType, SparkseeUtils.FIRST_NAME);
int lastNameAttr    = SparkseeUtils.getAttribute(personType, SparkseeUtils.LAST_NAME);
int countryNameAttr = SparkseeUtils.getAttribute(locationType,     SparkseeUtils.NAME);
int companyNameAttr = SparkseeUtils.getAttribute(organisationType, SparkseeUtils.NAME);
int workFromAttr    = SparkseeUtils.getAttribute(workAtType,       SparkseeUtils.WORK_FROM);


Objects friends = SparkseeUtils.getKnownPeople(graph, v, personId, 2);

v.setStringVoid(countryName);
long countryOID    = graph.findObject(countryNameAttr, v);
Objects companies  = graph.neighbors(countryOID, isLocatedInType, EdgesDirection.Ingoing);
Objects candidates        = graph.explode(friends,   workAtType, EdgesDirection.Outgoing);
Objects candidatesCompany = graph.explode(companies, workAtType, EdgesDirection.Ingoing);
companies.close();
friends.close();

candidates.intersection(candidatesCompany);
candidatesCompany.close();

v.setInteger(workFromYear);
Objects candidatesDate = graph.select(workFromAttr, Condition.LessThan, v);
candidates.intersection(candidatesDate);
candidatesDate.close();

//search matches
ObjectsIterator iterator = candidates.iterator();
while(iterator.hasNext()) {
    long oid = iterator.next();
    EdgeData data = graph.getEdgeData(oid);

    graph.getAttribute(data.getTail(), personIdAttr, v);
    long friendId = v.getLong();
    graph.getAttribute(data.getTail(), firstNameAttr, v);
    String firstName = v.getString();
    graph.getAttribute(data.getTail(), lastNameAttr, v);
    String lastName = v.getString();
    graph.getAttribute(data.getHead(), companyNameAttr, v);
    String organizationName = v.getString();
    graph.getAttribute(oid, workFromAttr, v);
    int friendWorkFrom = v.getInteger();

    result.add(new LdbcQuery11Result(friendId, firstName, lastName,
    organizationName, friendWorkFrom));
}
iterator.close();
candidates.close();

Collections.sort(result, new Comparator<LdbcQuery11Result>() {
    public int compare(LdbcQuery11Result r1, LdbcQuery11Result r2) {
        Integer date = r1.organizationWorkFromYear();
        // ascending by the start date
        int rc = date.compareTo(r2.organizationWorkFromYear()) ; 
        if (rc == 0) {
            Long id = r1.personId();
            // ascending by Person identifier
            rc = id.compareTo(r2.personId()); 
            if (rc == 0) {
                // Organization name descending
                rc = r2.organizationName().compareTo(r1.organizationName()); 
            }
        }
        return rc;
    }
});

sess.close();
return result.subList(0, Math.min(limit, result.size()));

\end{verbatim}
}

\subsection{Query 12}

{\footnotesize
\begin{verbatim}
List<LdbcQuery12Result> result = new ArrayList<LdbcQuery12Result>();
Graph graph = sess.getGraph();
Value v = new Value();

int personType       = SparkseeUtils.getType(SparkseeUtils.PERSON);
int postType         = SparkseeUtils.getType(SparkseeUtils.POST);
int tagClassType     = SparkseeUtils.getType(SparkseeUtils.TAGCLASS);
int tagType          = SparkseeUtils.getType(SparkseeUtils.TAG);
int hasTagType       = SparkseeUtils.getType(SparkseeUtils.HAS_TAG);
int hasCreatorType   = SparkseeUtils.getType(SparkseeUtils.HAS_CREATOR);
int replyOfType      = SparkseeUtils.getType(SparkseeUtils.REPLY_OF);
int isSubclassOfType = SparkseeUtils.getType(SparkseeUtils.IS_SUBCLASS_OF);
int hasTypeType      = SparkseeUtils.getType(SparkseeUtils.HAS_TYPE);

int personIdAttr     = SparkseeUtils.getAttribute(personType,   SparkseeUtils.ID);
int firstNameAttr    = SparkseeUtils.getAttribute(personType,   SparkseeUtils.FIRST_NAME);
int lastNameAttr     = SparkseeUtils.getAttribute(personType,   SparkseeUtils.LAST_NAME);
int tagClassNameAttr = SparkseeUtils.getAttribute(tagClassType, SparkseeUtils.NAME);
int tagNameAttr      = SparkseeUtils.getAttribute(tagType,      SparkseeUtils.NAME);


Objects friends = SparkseeUtils.getKnownPeople(graph, v, personId, 1);

v.setString(tagClassName);
long classOID = graph.findObject(tagClassNameAttr, v);
Objects tagClasses  = graph.neighbors(classOID, isSubclassOfType, EdgesDirection.Ingoing);

long size = 0;
while (size != tagClasses.count()) {
    size = tagClasses.count();
    Objects subClasses = graph.neighbors(tagClasses, 
                                         isSubclassOfType, 
                                         EdgesDirection.Ingoing);
    tagClasses.union(subClasses);
    subClasses.close();
}
tagClasses.add(classOID);
Objects messageTags = graph.neighbors(tagClasses,  hasTypeType, EdgesDirection.Ingoing);
Objects postOfTags  = graph.neighbors(messageTags, hasTagType,  EdgesDirection.Ingoing);
Objects posts = graph.select(postType); // Post only
postOfTags.intersection(posts);
posts.close();
tagClasses.close();


List<Long> tagList = new ArrayList<Long>();
Map<Long, String> tagMap = new HashMap<Long, String>();
ObjectsIterator tagIterator = messageTags.iterator();
while (tagIterator.hasNext()) {
    long oid = tagIterator.next();
    tagList.add(oid);
    graph.getAttribute(oid, tagNameAttr, v);
    tagMap.put(oid, v.getString());
}
tagIterator.close();
messageTags.close();

ObjectsIterator iterator = friends.iterator();
while(iterator.hasNext()) {
    long oid = iterator.next();

    Objects comments = graph.neighbors(oid, hasCreatorType, EdgesDirection.Ingoing);
    Objects replies  = graph.explode(comments, replyOfType, EdgesDirection.Ingoing);
    replies.intersection(postOfTags);
    comments.close();

    Set<String> tagNames = new HashSet<String>();
    if (replies.size() != 0) {
        Objects tagReplies  = graph.neighbors(replies, hasTagType, EdgesDirection.Ingoing);
        for (long tagOid : tagList) {
            if (tagReplies.exists(tagOid)) {
                tagNames.add(tagMap.get(tagOid));
            }
        }
        tagReplies.close();
    }

    graph.getAttribute(oid, personIdAttr, v);
    long friendId = v.getLong();
    graph.getAttribute(oid, firstNameAttr, v);
    String firstName = v.getString();
    graph.getAttribute(oid, lastNameAttr, v);
    String lastName = v.getString();

    result.add(new LdbcQuery12Result(friendId, firstName, lastName,
    tagNames, (int)replies.size()));
    replies.close();
}
iterator.close();
friends.close();
postOfTags.close();

Collections.sort(result, new Comparator<LdbcQuery12Result>() {
    public int compare(LdbcQuery12Result r1, LdbcQuery12Result r2) {
        Integer date = r2.replyCount();
        int rc = date.compareTo(r1.replyCount()); // descending by Comment count
        if (rc == 0) {
            Long id = r1.personId();
            rc = id.compareTo(r2.personId()); // ascending by Person identifier
        }
        return rc;
    }
});

sess.close();
return result.subList(0, Math.min(limit, result.size()));
\end{verbatim}
}

\subsection{Query 13}

{\footnotesize
\begin{verbatim}
List<LdbcQuery13Result> result = new ArrayList<LdbcQuery13Result>();

if (person1Id == person2Id) {
    result.add(new LdbcQuery13Result(0));
    sess.close();
    return result;
} 

Graph graph = sess.getGraph();
Value v = new Value();

int personType   = SparkseeUtils.getType(SparkseeUtils.PERSON);
int knowType     = SparkseeUtils.getType(SparkseeUtils.KNOWS);
int personIdAttr = SparkseeUtils.getAttribute(personType, SparkseeUtils.ID);

long person1OID = graph.findObject(personIdAttr, v.setLong(person1Id));
long person2OID = graph.findObject(personIdAttr, v.setLong(person2Id));

int length = PATH_NOT_FOUND;
try {
    @SuppressWarnings("resource")
    SinglePairShortestPathBFS shortestPath = new SinglePairShortestPathBFS(sess, person1OID, person2OID);
    shortestPath.addNodeType(personType);
    shortestPath.addEdgeType(knowType, EdgesDirection.Outgoing);
    shortestPath.run();

    length = shortestPath.exists() ? (int) shortestPath.getCost() : PATH_NOT_FOUND;
} catch (Exception e) {
}
result.add(new LdbcQuery13Result(length));
sess.close();
return result;
\end{verbatim}
}

\subsection{Query 14}

{\footnotesize
\begin{verbatim}
List<LdbcQuery14Result> result = new ArrayList<LdbcQuery14Result>();

Graph graph = sess.getGraph();
Value v = new Value();

int personType     = SparkseeUtils.getType(SparkseeUtils.PERSON);
int knowsType      = SparkseeUtils.getType(SparkseeUtils.KNOWS);
int postType       = SparkseeUtils.getType(SparkseeUtils.POST);
int commentType    = SparkseeUtils.getType(SparkseeUtils.COMMENT);
int replyOfType    = SparkseeUtils.getType(SparkseeUtils.REPLY_OF);
int hasCreatorType = SparkseeUtils.getType(SparkseeUtils.HAS_CREATOR);

int personIdAttr = SparkseeUtils.getAttribute(personType, SparkseeUtils.ID);
int weightAttr = graph.newSessionAttribute(knowsType, DataType.Double, AttributeKind.Basic);

if (person1Id == person2Id) {
    List<Long> path = new ArrayList<Long>();
    path.add(person1Id);
    result.add(new LdbcQuery14Result(path, 0.0));
    sess.close();
    return result;
}

// search all paths
long person1OID = graph.findObject(personIdAttr, v.setLong(person1Id));
long person2OID = graph.findObject(personIdAttr, v.setLong(person2Id));

int targetLength = 1;
Objects people = graph.neighbors(person1OID, knowsType, EdgesDirection.Outgoing);
int size = 0;
while (size != people.size() && !people.exists(person2OID)) {
size =  people.size();
Objects morePeople = graph.neighbors(people, knowsType, EdgesDirection.Outgoing);
people.union(morePeople);
targetLength++;
morePeople.close();
}

boolean itExists = people.exists(person2OID);
people.close();
if (!itExists) {
    targetLength = -1;
    sess.close();
    return result;
}

List<Objects> steps = new ArrayList<Objects>(targetLength);
steps.add(graph.neighbors(person1OID, knowsType, EdgesDirection.Outgoing));
for (int i = 1; i <= targetLength; i++) {
    steps.add(graph.neighbors(steps.get(i-1), knowsType, EdgesDirection.Outgoing));
}

List<Objects> backsteps = new ArrayList<Objects>(targetLength);
backsteps.add(graph.neighbors(person2OID, knowsType, EdgesDirection.Ingoing));
for (int i = 1; i <= targetLength; i++) {
    backsteps.add(graph.neighbors(backsteps.get(i-1), knowsType, EdgesDirection.Ingoing));
}

for (int i = 0; i < targetLength; i++) {
    steps.get(i).intersection(backsteps.get(targetLength-i));
}

for (Objects objs : backsteps) {
    objs.close();
}

List<Long> path = new ArrayList<Long>();
List<List<Long>> paths = new ArrayList<List<Long>>();
searchPaths(graph, targetLength, 0, person2OID, person1OID, knowsType, path, paths, steps);

for (Objects objs : steps) {
    objs.close();
}

Objects posts    = graph.select(postType);
Objects comments = graph.select(commentType);
for (int i = 0; i < paths.size(); i++) {
    List<Long> pathId = new ArrayList<Long>();
    for (int j = 0; j < paths.get(i).size(); j++) {
        graph.getAttribute(paths.get(i).get(j), personIdAttr, v);
        pathId.add(v.getLong());
    }
    double weight = 0.0;
    for (int j = 1; j < paths.get(i).size(); j++) {
        long tail = paths.get(i).get(j-1);
        long head = paths.get(i).get(j);
        long edgeOid = graph.findEdge(knowsType, tail, head);
        graph.getAttribute(edgeOid, weightAttr, v);
        if (v.isNull()) {

            Objects messages1 = graph.neighbors(tail, hasCreatorType, 
                                                EdgesDirection.Ingoing);
            Objects messages2 = graph.neighbors(head, hasCreatorType, 
                                                EdgesDirection.Ingoing);
            Objects replies1  = graph.neighbors(messages1, replyOfType, 
                                                EdgesDirection.Outgoing);
            Objects replies2  = graph.neighbors(messages2, replyOfType, 
                                                EdgesDirection.Outgoing);
            replies1.intersection(messages2);
            replies2.intersection(messages1);
            replies1.union(replies2);
            messages1.close();
            messages2.close();
            replies2.close();

            Objects replyOfPost = Objects.combineIntersection(replies1, posts);
            replies1.intersection(comments);
            Double score1 = new Double(replyOfPost.count());
            Double score2 = new Double(replies1.count() / 2.0);
            replies1.close();
            replyOfPost.close();

            graph.setAttribute(edgeOid, weightAttr, v.setDouble(score1 + score2));
        }
        weight += v.getDouble();
    }
    result.add(new LdbcQuery14Result(pathId, weight));
}
posts.close();
comments.close();

Collections.sort(result, new Comparator<LdbcQuery14Result>() {
    public int compare(LdbcQuery14Result r1, LdbcQuery14Result r2) {
        Double weight = r2.pathWeight();
        // descending by path weight
        return weight.compareTo(r1.pathWeight()); 
    }
});

sess.close();
return result;

\end{verbatim}
}

%%%%%%%%%%%%%%%%%%%%%%%%%%%%%%%%%%%%%%%%%%%%%%%%%%%%%%%%%%%%%%%%%%%%%%%%%%%%%%
%%%%%%%%%%%%%%%%%%%%%%%%%%%%%%%%%%%%%%%%%%%%%%%%%%%%%%%%%%%%%%%%%%%%%%%%%%%%%%
%%%%%%%%%%%%%%%%%%%%%%%%%%%%%%%%%%%%%%%%%%%%%%%%%%%%%%%%%%%%%%%%%%%%%%%%%%%%%%



\chapter{Scale Factor Statistics}
\section{Scale Factor Statistics}\label{appendix:scale_factors}
\subsection{Scale Factor 1}

\begin{table}[H]
\centering
    \begin{tabular}{|c|c|c|c|}
    \hline
    Query Type & freq & Query Type & freq \\ 
    \hline
    \hline
    Query 1 & 26 & Query 8 & 45 \\ 
    \hline       
    Query 2 & 37 & Query 9 & 157 \\  
    \hline        
    Query 3 & 69 & Query 10 & 30 \\ 
    \hline        
    Query 4 & 36 & Query 11 & 16 \\ 
    \hline        
    Query 5 & 57 & Query 12 & 44 \\ 
    \hline        
    Query 6 & 129 & Query 13 & 19 \\  
    \hline        
    Query 7 & 87 & Query 14 & 49 \\ 
    \hline
    \end{tabular}
    \caption{Frequenceis for each query type for SF1.}
    \label{table:freqs_sf1}
\end{table}

\begin{table}[H]
    \centering
    \begin{tabular} {| l | c | c |}
        \hline
        \textbf{Entity} & \textbf{Num Entities} & \textbf{Bytes} \\
        \hline
        \hline
        comment & 2343952 & 254723836 \\
        \hline
        forum & 110202 & 6548409 \\
        \hline
        organisation & 7996 & 813270 \\
        \hline
        person & 11000 & 990357 \\
        \hline
        place & 1466 & 83667 \\
        \hline
        post & 1214766 & 138430549 \\
        \hline
        tag & 16080 & 1122429 \\
        \hline
        tagclass & 71 & 3946 \\
        \hline
        \hline
        \textbf{Relation} & \textbf{Num Relations} & \textbf{Bytes} \\
        \hline
        \hline
        comment\_hasCreator\_person & 2343952 & 63507355 \\
        \hline
        comment\_hasTag\_tag & 3069162 & 57501504 \\
        \hline
        comment\_isLocatedIn\_place & 2343952 & 39543099 \\
        \hline
        comment\_replyOf\_comment & 1187815 & 31674987 \\
        \hline
        comment\_replyOf\_post & 1156137 & 30828349 \\
        \hline
        forum\_containerOf\_post & 1214766 & 32211087 \\
        \hline
        forum\_hasMember\_person & 3260578 & 159205747 \\
        \hline
        forum\_hasModerator\_person & 110202 & 3017841 \\
        \hline
        forum\_hasTag\_tag & 355354 & 6527532 \\
        \hline
        organisation\_isLocatedIn\_place & 7996 & 79310 \\
        \hline
        person\_isLocatedIn\_place & 11000 & 196342 \\
        \hline
        person\_hasInterest\_tag & 256152 & 5120644 \\
        \hline
        person\_knows\_person & 452622 & 22659548 \\
        \hline
        person\_likes\_comment & 1649394 & 80566053 \\
        \hline
        person\_likes\_post & 1170372 & 57185940 \\
        \hline
        person\_studyAt\_organisation & 8820 & 221093 \\
        \hline
        person\_workAt\_organisation & 23969 & 581247 \\
        \hline
        place\_isPartOf\_place & 1460 & 11965 \\
        \hline
        post\_hasCreator\_person & 1214766 & 33212920 \\
        \hline
        post\_hasTag\_tag & 789735 & 14621607 \\
        \hline
        post\_isLocatedIn\_place & 1214766 & 20529353 \\
        \hline
        tag\_hasType\_tagclass & 16080 & 163348 \\
        \hline
        tagclass\_isSubclassOf\_tagclass & 70 & 616 \\
        \hline
        \hline
        \textbf{Property Files} & \textbf{Num Properties} & \textbf{Bytes} \\
        \hline
        \hline
        person\_email\_emailaddress & 18602 & 831575 \\
        \hline
        person\_speaks\_language & 24204 & 437214 \\
        \hline
        \hline
        \textbf{Total Entities} & \textbf{Total Relations} & \textbf{Total Bytes} \\
        \hline
        \hline
        3705533 & 21859120 & 1063152739 \\
        \hline
    \end{tabular}
    \caption{General statistics for SF 1}
\end{table}


\begin{table}[H]
        \centering
        \begin{tabular}{|c||r|r|r|r|}
            \hline    \multicolumn{5}{|c|}{SF = 1 }  \\
            \hline   \textbf{Clustering Coef.} &   \multicolumn{4}{|c|}{0.0484} \\
            \hline & \textbf{Min} & \textbf{Max} & \textbf{Mean} & \textbf{Median}   \\
            \hline  \textbf{\#comments/user}  &1 &  6002 & 224 & 82 \\
            \hline  \textbf{\#posts/user}  &1 &  912 & 123 & 66 \\
            \hline  \textbf{\#friends/user}  &1 &  540 & 41 & 22 \\
            \hline  \textbf{\#likes/user}  &1 &  2725 & 260 & 171 \\
            \hline
        \end{tabular}
        \caption{Detail statistics for SF 1}
\end{table}

\begin{figure}[H]
\begin{center}
  \subfigure[{\small Cumulative number of friends.}]{\label{numFriendsCumm}\includegraphics[width=0.32\textwidth,clip]{figures/scalefactor/numFriendsCumm1}}%
  \subfigure[{\small Histogram \#posts per user.}]{\label{numPostPerUsers}\includegraphics[width=0.32\textwidth,clip]{figures/scalefactor/numPostsUserHist1}}%\\[1ex]
  \subfigure[{\small Histogram \#comments per user.}]{\label{numComPerUsers}\includegraphics[width=0.32\textwidth,clip]{figures/scalefactor/numCommentsUserHist1}}
  \caption{Data distributions for SF 1}
  \label{fig:datadistSF1}
\end{center}
\end{figure}


\subsection{Scale Factor 3}

\begin{table}[H]
\centering
    \begin{tabular}{|c|c|c|c|}
    \hline
    Query Type & freq & Query Type & freq \\ 
    \hline
    \hline
    Query 1 & 26 & Query 8 & 27 \\ 
    \hline       
    Query 2 & 37 & Query 9 & 209 \\  
    \hline        
    Query 3 & 79 & Query 10 & 32 \\ 
    \hline       
    Query 4 & 36 & Query 11 & 17 \\ 
    \hline        
    Query 5 & 61 & Query 12 & 44 \\ 
    \hline        
    Query 6 & 172 & Query 13 & 19 \\  
    \hline        
    Query 7 & 72 & Query 14 & 49 \\ 
    \hline
    \end{tabular}
    \caption{Frequenceis for each query type for SF3.}
    \label{table:freqs_sf3}
\end{table}

\begin{table}[H]
    \centering
    \begin{tabular} {| l | c | c |}
        \hline
        \textbf{Entity} & \textbf{Num Entities} & \textbf{Bytes} \\
        \hline
        \hline
        comment & 7135636 & 776534811 \\
        \hline
        forum & 272268 & 16231309 \\
        \hline
        organisation & 7996 & 813270 \\
        \hline
        person & 27000 & 2431528 \\
        \hline
        place & 1466 & 83667 \\
        \hline
        post & 3140119 & 374416646 \\
        \hline
        tag & 16080 & 1122429 \\
        \hline
        tagclass & 71 & 3946 \\
        \hline
        \hline
        \textbf{Relation} & \textbf{Num Relations} & \textbf{Bytes} \\
        \hline
        \hline
        comment\_hasCreator\_person & 7135636 & 194770123 \\
        \hline
        comment\_hasTag\_tag & 9264389 & 174656230 \\
        \hline
        comment\_isLocatedIn\_place & 7135636 & 121173303 \\
        \hline
        comment\_replyOf\_comment & 3619711 & 97338366 \\
        \hline
        comment\_replyOf\_post & 3515925 & 94545033 \\
        \hline
        forum\_containerOf\_post & 3140119 & 83915474 \\
        \hline
        forum\_hasMember\_person & 9939453 & 486936117 \\
        \hline
        forum\_hasModerator\_person & 272268 & 7495375 \\
        \hline
        forum\_hasTag\_tag & 873831 & 16205018 \\
        \hline
        organisation\_isLocatedIn\_place & 7996 & 79310 \\
        \hline
        person\_isLocatedIn\_place & 27000 & 482925 \\
        \hline
        person\_hasInterest\_tag & 628563 & 12575921 \\
        \hline
        person\_knows\_person & 1370174 & 68746822 \\
        \hline
        person\_likes\_comment & 5555074 & 272259351 \\
        \hline
        person\_likes\_post & 3629288 & 177882573 \\
        \hline
        person\_studyAt\_organisation & 21574 & 541636 \\
        \hline
        person\_workAt\_organisation & 58843 & 1428856 \\
        \hline
        place\_isPartOf\_place & 1460 & 11965 \\
        \hline
        post\_hasCreator\_person & 3140119 & 86258384 \\
        \hline
        post\_hasTag\_tag & 2384629 & 44446829 \\
        \hline
        post\_isLocatedIn\_place & 3140119 & 53352987 \\
        \hline
        tag\_hasType\_tagclass & 16080 & 163348 \\
        \hline
        tagclass\_isSubclassOf\_tagclass & 70 & 616 \\
        \hline
        \hline
        \textbf{Property Files} & \textbf{Num Properties} & \textbf{Bytes} \\
        \hline
        \hline
        person\_email\_emailaddress & 45573 & 2041123 \\
        \hline
        person\_speaks\_language & 59467 & 1076428 \\
        \hline
        \hline
        \textbf{Total Entities} & \textbf{Total Relations} & \textbf{Total Bytes} \\
        \hline
        \hline
         10600636 & 64877957 & 3170021719 \\
        \hline
    \end{tabular}
    \caption{General statistics for SF 3}
\end{table}

\begin{table}[H]
    \centering
\begin{tabular}{|c||r|r|r|r|}
\hline    \multicolumn{5}{|c|}{SF = 3 }  \\
\hline   \textbf{Clustering Coef.} &   \multicolumn{4}{|c|}{0.0456} \\
\hline & \textbf{Min} & \textbf{Max} & \textbf{Mean} & \textbf{Median}   \\
\hline  \textbf{\#comments/user}  &1 &  6631 & 275 & 102 \\
\hline  \textbf{\#posts/user}  &1 &  1096 & 128 & 72 \\
\hline  \textbf{\#friends/user}  &1 &  569 & 51 & 28 \\
\hline  \textbf{\#likes/user}  &1 &  3057 & 344 & 231 \\
\hline
\end{tabular}
\caption{Detail statistics for SF 3}
\end{table}

\begin{figure}[H]
\begin{center}
  \subfigure[{\small Cumulative number of friends.}]{\label{numFriendsCumm}\includegraphics[width=0.32\textwidth,clip]{figures/scalefactor/numFriendsCumm3}}%
  \subfigure[{\small Histogram \#posts per user.}]{\label{numPostPerUsers}\includegraphics[width=0.32\textwidth,clip]{figures/scalefactor/numPostsUserHist3}}%\\[1ex]
  \subfigure[{\small Histogram \#comments per user.}]{\label{numComPerUsers}\includegraphics[width=0.32\textwidth,clip]{figures/scalefactor/numCommentsUserHist3}}
  \caption{Data distributions for SF 3}
  \label{fig:datadistSF3}
\end{center}
\end{figure}

\subsection{Scale Factor 10}

\begin{table}[H]
\centering
    \begin{tabular}{|c|c|c|c|}
    \hline
    Query Type & freq & Query Type & freq \\ 
    \hline
    \hline
    Query 1 & 26 & Query 8 & 15 \\ 
    \hline       
    Query 2 & 37 & Query 9 & 287 \\  
    \hline        
    Query 3 & 92 & Query 10 & 35 \\ 
    \hline       
    Query 4 & 36 & Query 11 & 19 \\ 
    \hline        
    Query 5 & 66 & Query 12 & 44 \\ 
    \hline        
    Query 6 & 236 & Query 13 & 19 \\  
    \hline        
    Query 7 & 54 & Query 14 & 49 \\ 
    \hline
    \end{tabular}
    \caption{Frequenceis for each query type for SF10.}
    \label{table:freqs_sf10}
\end{table}

\begin{table}[H]
    \centering
    \begin{tabular} {| l | c | c |}
        \hline
        \textbf{Entity} & \textbf{Num Entities} & \textbf{Bytes} \\
        \hline
        \hline
        comment & 24271888 & 2648214861 \\
        \hline
        forum & 729153 & 43643724 \\
        \hline
        organisation & 7996 & 813324 \\
        \hline
        person & 73000 & 6570890 \\
        \hline
        place & 1466 & 83721 \\
        \hline
        post & 8915649 & 1126585578 \\
        \hline
        tag & 16080 & 1122468 \\
        \hline
        tagclass & 71 & 3985 \\
        \hline
        \hline
        \textbf{Relation} & \textbf{Num Relations} & \textbf{Bytes} \\
        \hline
        \hline
        comment\_hasCreator\_person & 24271888 & 669164047 \\
        \hline
        comment\_hasTag\_tag & 31753457 & 605414570 \\
        \hline
        comment\_isLocatedIn\_place & 24271888 & 418145702 \\
        \hline
        comment\_replyOf\_comment & 12306670 & 336987410 \\
        \hline
        comment\_replyOf\_post & 11965218 & 327636871 \\
        \hline
        forum\_containerOf\_post & 8915649 & 242973393 \\
        \hline
        forum\_hasMember\_person & 33883607 & 1670125108 \\
        \hline
        forum\_hasModerator\_person & 729153 & 20284418 \\
        \hline
        forum\_hasTag\_tag & 2369727 & 44544367 \\
        \hline
        organisation\_isLocatedIn\_place & 7996 & 79388 \\
        \hline
        person\_isLocatedIn\_place & 73000 & 1305804 \\
        \hline
        person\_hasInterest\_tag & 1713574 & 34283207 \\
        \hline
        person\_knows\_person & 4654416 & 233569942 \\
        \hline
        person\_likes\_comment & 21418614 & 1054924693 \\
        \hline
        person\_likes\_post & 12661782 & 623979230 \\
        \hline
        person\_studyAt\_organisation & 58429 & 1467151 \\
        \hline
        person\_workAt\_organisation & 158961 & 3860488 \\
        \hline
        place\_isPartOf\_place & 1460 & 12022 \\
        \hline
        post\_hasCreator\_person & 8915649 & 247527557 \\
        \hline
        post\_hasTag\_tag & 8216364 & 154770790 \\
        \hline
        post\_isLocatedIn\_place & 8915649 & 154055825 \\
        \hline
        tag\_hasType\_tagclass & 16080 & 163408 \\
        \hline
        tagclass\_isSubclassOf\_tagclass & 70 & 691 \\
        \hline
        \hline
        \textbf{Property Files} & \textbf{Num Properties} & \textbf{Bytes} \\
        \hline
        \hline
        person\_email\_emailaddress & 124555 & 5574325 \\
        \hline
        person\_speaks\_language & 160779 & 2910238 \\
        \hline
        \hline
        \textbf{Total Entities} & \textbf{Total Relations} & \textbf{Total Bytes} \\
        \hline
        \hline
         34015303 & 217279301 & 10680799196 \\
        \hline
    \end{tabular}
    \caption{General statistics for SF 10}
\end{table}

\begin{table}[H]
\centering
\begin{tabular}{|c||r|r|r|r|}
\hline    \multicolumn{5}{|c|}{SF = 3 }  \\
\hline   \textbf{Clustering Coef.} &   \multicolumn{4}{|c|}{0.0456} \\
\hline & \textbf{Min} & \textbf{Max} & \textbf{Mean} & \textbf{Median}   \\
\hline  \textbf{\#comments/user}  &1 &  6631 & 275 & 102 \\
\hline  \textbf{\#posts/user}  &1 &  1096 & 128 & 72 \\
\hline  \textbf{\#friends/user}  &1 &  569 & 51 & 28 \\
\hline  \textbf{\#likes/user}  &1 &  3057 & 344 & 231 \\
\hline
\end{tabular}
\caption{Detail statistics for SF 10}
\end{table}

\begin{figure}[H]
\begin{center}
  \subfigure[{\small Cumulative number of friends.}]{\label{numFriendsCumm}\includegraphics[width=0.32\textwidth,clip]{figures/scalefactor/numFriendsCumm10}}%
  \subfigure[{\small Histogram \#posts per user.}]{\label{numPostPerUsers}\includegraphics[width=0.32\textwidth,clip]{figures/scalefactor/numPostsUserHist10}}%\\[1ex]
  \subfigure[{\small Histogram \#comments per user.}]{\label{numComPerUsers}\includegraphics[width=0.32\textwidth,clip]{figures/scalefactor/numCommentsUserHist10}}
  \caption{Data distributions for SF 10}
  \label{fig:datadistSF10}
\end{center}
\end{figure}

\subsection{Scale Factor 30}

\begin{table}[H]
\centering
    \begin{tabular}{|c|c|c|c|}
    \hline
    Query Type & freq & Query Type & freq \\ 
    \hline
    \hline
    Query 1 & 26 & Query 8 & 9 \\ 
    \hline       
    Query 2 & 37 & Query 9 & 384 \\  
    \hline        
    Query 3 & 106 & Query 10 & 37 \\ 
    \hline       
    Query 4 & 36 & Query 11 & 20 \\ 
    \hline        
    Query 5 & 72 & Query 12 & 44 \\ 
    \hline        
    Query 6 & 316 & Query 13 & 19 \\  
    \hline        
    Query 7 & 48 & Query 14 & 49 \\ 
    \hline
    \end{tabular}
    \caption{Frequenceis for each query type for SF30.}
    \label{table:freqs_sf30}
\end{table}

\begin{table}[H]
    \centering
    \begin{tabular} {| l | c | c |}
        \hline
        \textbf{Entity} & \textbf{Num Entities} & \textbf{Bytes} \\
        \hline
        \hline
        comment & 73590941 & 8083989095 \\
        \hline
        forum & 1842141 & 111539981 \\
        \hline
        organisation & 7996 & 813396 \\
        \hline
        person & 184000 & 16572878 \\
        \hline
        place & 1466 & 83793 \\
        \hline
        post & 23765756 & 3155561666 \\
        \hline
        tag & 16080 & 1122520 \\
        \hline
        tagclass & 71 & 4037 \\
        \hline
        \hline
        \textbf{Relation} & \textbf{Num Relations} & \textbf{Bytes} \\
        \hline
        \hline
        comment\_hasCreator\_person & 73590941 & 2088295129 \\
        \hline
        comment\_hasTag\_tag & 96053813 & 1903298754 \\
        \hline
        comment\_isLocatedIn\_place & 73590941 & 1320854361 \\
        \hline
        comment\_replyOf\_comment & 37324357 & 1075860096 \\
        \hline
        comment\_replyOf\_post & 36266584 & 1045376200 \\
        \hline
        forum\_containerOf\_post & 23765756 & 679608557 \\
        \hline
        forum\_hasMember\_person & 103901443 & 5196088120 \\
        \hline
        forum\_hasModerator\_person & 1842141 & 52580681 \\
        \hline
        forum\_hasTag\_tag & 5976729 & 116509043 \\
        \hline
        organisation\_isLocatedIn\_place & 7996 & 79492 \\
        \hline
        person\_isLocatedIn\_place & 184000 & 3297409 \\
        \hline
        person\_hasInterest\_tag & 4318588 & 86533802 \\
        \hline
        person\_knows\_person & 14212356 & 714378938 \\
        \hline
        person\_likes\_comment & 71641419 & 3584484467 \\
        \hline
        person\_likes\_post & 39694513 & 1986127459 \\
        \hline
        person\_studyAt\_organisation & 147005 & 3695367 \\
        \hline
        person\_workAt\_organisation & 401356 & 9761198 \\
        \hline
        place\_isPartOf\_place & 1460 & 12098 \\
        \hline
        post\_hasCreator\_person & 23765756 & 677464115 \\
        \hline
        post\_hasTag\_tag & 24931521 & 488840146 \\
        \hline
        post\_isLocatedIn\_place & 23765756 & 426900332 \\
        \hline
        tag\_hasType\_tagclass & 16080 & 163488 \\
        \hline
        tagclass\_isSubclassOf\_tagclass & 70 & 791 \\
        \hline
        \hline
        \textbf{Property Files} & \textbf{Num Properties} & \textbf{Bytes} \\
        \hline
        \hline
        person\_email\_emailaddress & 312925 & 14030700 \\
        \hline
        person\_speaks\_language & 405403 & 7353001 \\
        \hline
        \hline
        \textbf{Total Entities} & \textbf{Total Relations} & \textbf{Total Bytes} \\
        \hline
        \hline
         99408451 & 655400581 & 32851281110 \\
        \hline
    \end{tabular}
    \caption{General statistics for SF 30}
\end{table}

\begin{table}[H]
    \centering
\begin{tabular}{|c||r|r|r|r|}
\hline    \multicolumn{5}{|c|}{SF = 30 }  \\
\hline   \textbf{Clustering Coef.} &   \multicolumn{4}{|c|}{0.0439} \\
                            \hline & \textbf{Min} & \textbf{Max} & \textbf{Mean} & \textbf{Median}   \\
 \hline  \textbf{\#comments/user}  &1 &  7592 & 413 & 155 \\
    \hline  \textbf{\#posts/user}  &1 &  1412 & 139 & 83 \\
  \hline  \textbf{\#friends/user}  &1 &  625 & 77 & 43 \\
    \hline  \textbf{\#likes/user}  &1 &  3828 & 610 & 420 \\
\hline
\end{tabular}
\caption{Detail statistics for SF 30}
\end{table}

\begin{figure}[H]
\begin{center}
  \subfigure[{\small Cumulative number of friends.}]{\label{numFriendsCumm}\includegraphics[width=0.32\textwidth,clip]{figures/scalefactor/numFriendsCumm30}}%
  \subfigure[{\small Histogram \#posts per user.}]{\label{numPostPerUsers}\includegraphics[width=0.32\textwidth,clip]{figures/scalefactor/numPostsUserHist30}}%\\[1ex]
  \subfigure[{\small Histogram \#comments per user.}]{\label{numComPerUsers}\includegraphics[width=0.32\textwidth,clip]{figures/scalefactor/numCommentsUserHist30}}
  \caption{Data distributions for SF 30}
  \label{fig:datadistSF30}
\end{center}
\end{figure}

\subsection{Scale Factor 100}

\begin{table}[H]
\centering
    \begin{tabular}{|c|c|c|c|}
    \hline
    Query Type & freq & Query Type & freq \\ 
    \hline
    \hline
    Query 1 & 26 & Query 8 & 5 \\ 
    \hline       
    Query 2 & 37 & Query 9 & 527 \\  
    \hline        
    Query 3 & 123 & Query 10 & 40 \\ 
    \hline       
    Query 4 & 36 & Query 11 & 22  \\ 
    \hline        
    Query 5 & 78 & Query 12 & 44  \\ 
    \hline        
    Query 6 & 434 & Query 13 & 19  \\  
    \hline        
    Query 7 & 38 & Query 14 & 49  \\ 
    \hline
    \end{tabular}
    \caption{Frequenceis for each query type for SF100.}
    \label{table:freqs_sf100}
\end{table}

\begin{table}[H]
    \centering
    \begin{tabular} {| l | c | c |}
        \hline
        \textbf{Entity} & \textbf{Num Entities} & \textbf{Bytes} \\
        \hline
        \hline
        comment & 243266898 & 26732787716 \\
        \hline
        forum & 5002291 & 303107584 \\
        \hline
        organisation & 7996 & 813396 \\
        \hline
        person & 499000 & 44950237 \\
        \hline
        place & 1466 & 83793 \\
        \hline
        post & 68871360 & 9601082178 \\
        \hline
        tag & 16080 & 1122520 \\
        \hline
        tagclass & 71 & 4037 \\
        \hline
        \hline
        \textbf{Relation} & \textbf{Num Relations} & \textbf{Bytes} \\
        \hline
        \hline
        comment\_hasCreator\_person & 243266898 & 6923334782 \\
        \hline
        comment\_hasTag\_tag & 317369562 & 6310390486 \\
        \hline
        comment\_isLocatedIn\_place & 243266898 & 4380836100 \\
        \hline
        comment\_replyOf\_comment & 123386519 & 3571363911 \\
        \hline
        comment\_replyOf\_post & 119880379 & 3469854233 \\
        \hline
        forum\_containerOf\_post & 68871360 & 1977411509 \\
        \hline
        forum\_hasMember\_person & 341232279 & 17085982726 \\
        \hline
        forum\_hasModerator\_person & 5002291 & 143155976 \\
        \hline
        forum\_hasTag\_tag & 16195463 & 317441296 \\
        \hline
        organisation\_isLocatedIn\_place & 7996 & 79492 \\
        \hline
        person\_isLocatedIn\_place & 499000 & 8948068 \\
        \hline
        person\_hasInterest\_tag & 11692172 & 234436590 \\
        \hline
        person\_knows\_person & 46598276 & 2343165388 \\
        \hline
        person\_likes\_comment & 260701994 & 13062653343 \\
        \hline
        person\_likes\_post & 135205141 & 6773886764 \\
        \hline
        person\_studyAt\_organisation & 398560 & 10023920 \\
        \hline
        person\_workAt\_organisation & 1086037 & 26420132 \\
        \hline
        place\_isPartOf\_place & 1460 & 12098 \\
        \hline
        post\_hasCreator\_person & 68871360 & 1968125668 \\
        \hline
        post\_hasTag\_tag & 82466083 & 1623280287 \\
        \hline
        post\_isLocatedIn\_place & 68871360 & 1240297918 \\
        \hline
        tag\_hasType\_tagclass & 16080 & 163488 \\
        \hline
        tagclass\_isSubclassOf\_tagclass & 70 & 791 \\
        \hline
        \hline
        \textbf{Property Files} & \textbf{Num Properties} & \textbf{Bytes} \\
        \hline
        \hline
        person\_email\_emailaddress & 850804 & 38160557 \\
        \hline
        person\_speaks\_language & 1099440 & 19951911 \\
        \hline
        \hline
        \textbf{Total Entities} & \textbf{Total Relations} & \textbf{Total Bytes} \\
        \hline
        \hline
         317665162 & 2154887238 & 108213328895 \\
        \hline
    \end{tabular}
    \caption{General statistics for SF 100}
\end{table}

\begin{table}[H]
\centering
\begin{tabular}{|c||r|r|r|r|}
\hline    \multicolumn{5}{|c|}{SF = 100 }  \\
\hline   \textbf{Clustering Coef.} &   \multicolumn{4}{|c|}{0.0422} \\
                            \hline & \textbf{Min} & \textbf{Max} & \textbf{Mean} & \textbf{Median}   \\
 \hline  \textbf{\#comments/user}  &1 &  7465 & 502 & 190 \\
    \hline  \textbf{\#posts/user}  &1 &  1509 & 148 & 90 \\
  \hline  \textbf{\#friends/user}  &1 &  619 & 93 & 53 \\
    \hline  \textbf{\#likes/user}  &1 &  4312 & 799 & 556 \\
\hline
\end{tabular}
\caption{Detail statistics for SF 100}
\end{table}

\begin{figure}[H]
\begin{center}
  \subfigure[{\small Cumulative number of friends.}]{\label{numFriendsCumm}\includegraphics[width=0.32\textwidth,clip]{figures/scalefactor/numFriendsCumm100}}%
  \subfigure[{\small Histogram \#posts per user.}]{\label{numPostPerUsers}\includegraphics[width=0.32\textwidth,clip]{figures/scalefactor/numPostsUserHist100}}%\\[1ex]
  \subfigure[{\small Histogram \#comments per user.}]{\label{numComPerUsers}\includegraphics[width=0.32\textwidth,clip]{figures/scalefactor/numCommentsUserHist100}}
  \caption{Data distributions for SF 100}
  \label{fig:datadistSF10}
\end{center}
\end{figure}

\subsection{Scale Factor 300}

\begin{table}[H]
\centering
    \begin{tabular}{|c|c|c|c|}
    \hline
    Query Type & freq & Query Type & freq \\ 
    \hline
    \hline
    Query 1 & 26 & Query 8 & 3 \\ 
    \hline       
    Query 2 & 37 & Query 9 & 705 \\  
    \hline        
    Query 3 & 142 & Query 10 & 44 \\ 
    \hline       
    Query 4 & 36 & Query 11 &  24 \\ 
    \hline        
    Query 5 & 84 & Query 12 &  44 \\ 
    \hline        
    Query 6 & 580 & Query 13 &  19 \\  
    \hline        
    Query 7 & 32 & Query 14 &  49 \\ 
    \hline
    \end{tabular}
    \caption{Frequenceis for each query type for SF300.}
    \label{table:freqs_sf300}
\end{table}

\begin{table}[H]
    \centering
    \begin{tabular} {| l | c | c |}
        \hline
        \textbf{Entity} & \textbf{Num Entities} & \textbf{Bytes} \\
        \hline
        \hline
        comment & 710752235 & 78578510866 \\
        \hline
        forum & 12561079 & 769736017 \\
        \hline
        organisation & 7996 & 813396 \\
        \hline
        person & 1254000 & 113011768 \\
        \hline
        place & 1466 & 83793 \\
        \hline
        post & 182980982 & 26615002745 \\
        \hline
        tag & 16080 & 1122520 \\
        \hline
        tagclass & 71 & 4037 \\
        \hline
        \hline
        \textbf{Relation} & \textbf{Num Relations} & \textbf{Bytes} \\
        \hline
        \hline
        comment\_hasCreator\_person & 710752235 & 20740234727 \\
        \hline
        comment\_hasTag\_tag & 926124724 & 19010889474 \\
        \hline
        comment\_isLocatedIn\_place & 710752235 & 13268389734 \\
        \hline
        comment\_replyOf\_comment & 360517003 & 10910496465 \\
        \hline
        comment\_replyOf\_post & 350235232 & 10599470746 \\
        \hline
        forum\_containerOf\_post & 182980982 & 5473898610 \\
        \hline
        forum\_hasMember\_person & 995330706 & 50531158002 \\
        \hline
        forum\_hasModerator\_person & 12561079 & 368655691 \\
        \hline
        forum\_hasTag\_tag & 40653342 & 819806778 \\
        \hline
        organisation\_isLocatedIn\_place & 7996 & 79492 \\
        \hline
        person\_isLocatedIn\_place & 1254000 & 22520270 \\
        \hline
        person\_hasInterest\_tag & 29346263 & 589162363 \\
        \hline
        person\_knows\_person & 136219368 & 6857187354 \\
        \hline
        person\_likes\_comment & 820056009 & 41645511118 \\
        \hline
        person\_likes\_post & 404808353 & 20560829944 \\
        \hline
        person\_studyAt\_organisation & 1002380 & 25237062 \\
        \hline
        person\_workAt\_organisation & 2728559 & 66447988 \\
        \hline
        place\_isPartOf\_place & 1460 & 12098 \\
        \hline
        post\_hasCreator\_person & 182980982 & 5363521573 \\
        \hline
        post\_hasTag\_tag & 241151541 & 4898991661 \\
        \hline
        post\_isLocatedIn\_place & 182980982 & 3419470093 \\
        \hline
        tag\_hasType\_tagclass & 16080 & 163488 \\
        \hline
        tagclass\_isSubclassOf\_tagclass & 70 & 791 \\
        \hline
        \hline
        \textbf{Property Files} & \textbf{Num Properties} & \textbf{Bytes} \\
        \hline
        \hline
        person\_email\_emailaddress & 2140338 & 96111129 \\
        \hline
        person\_speaks\_language & 2763075 & 50221509 \\
        \hline
        \hline
        \textbf{Total Entities} & \textbf{Total Relations} & \textbf{Total Bytes} \\
        \hline
        \hline
         907573909 & 6292461581 & 321396753302 \\
        \hline
    \end{tabular}
    \caption{ General statistics for SF 300}
\end{table}

\begin{table}[H]
    \centering
\begin{tabular}{|c||r|r|r|r|}
\hline    \multicolumn{5}{|c|}{SF = 300 }  \\
\hline   \textbf{Clustering Coef.} &   \multicolumn{4}{|c|}{0.0411} \\
                            \hline & \textbf{Min} & \textbf{Max} & \textbf{Mean} & \textbf{Median}   \\
 \hline  \textbf{\#comments/user}  &1 &  8806 & 582 & 224 \\
    \hline  \textbf{\#posts/user}  &1 &  1501 & 155 & 97 \\
  \hline  \textbf{\#friends/user}  &1 &  620 & 109 & 63 \\
    \hline  \textbf{\#likes/user}  &1 &  4686 & 983 & 689 \\
\hline
\end{tabular}
\caption{Detail statistics for SF 300}
\end{table}

\begin{figure}[H]
\begin{center}
  \subfigure[{\small Cumulative number of friends.}]{\label{numFriendsCumm}\includegraphics[width=0.32\textwidth,clip]{figures/scalefactor/numFriendsCumm300}}%
  \subfigure[{\small Histogram \#posts per user.}]{\label{numPostPerUsers}\includegraphics[width=0.32\textwidth,clip]{figures/scalefactor/numPostsUserHist300}}%\\[1ex]
  \subfigure[{\small Histogram \#comments per user.}]{\label{numComPerUsers}\includegraphics[width=0.32\textwidth,clip]{figures/scalefactor/numCommentsUserHist300}}
  \caption{Data distributions for SF 300}
  \label{fig:datadistSF300}
\end{center}
\end{figure}

\subsection{Scale Factor 1000}

\begin{table}[H]
\centering
    \begin{tabular}{|c|c|c|c|}
    \hline
    Query Type & freq & Query Type & freq \\ 
    \hline
    \hline
    Query 1 & 26 & Query 8 & 1 \\ 
    \hline       
    Query 2 & 37 & Query 9 & 967 \\  
    \hline        
    Query 3 & 165 & Query 10 & 47 \\ 
    \hline       
    Query 4 & 36 & Query 11 &  26 \\ 
    \hline        
    Query 5 & 91 & Query 12 &  44 \\ 
    \hline        
    Query 6 & 796 & Query 13 &  19 \\  
    \hline        
    Query 7 & 25 & Query 14 &  49 \\ 
    \hline
    \end{tabular}
    \caption{Frequenceis for each query type for SF1000.}
    \label{table:freqs_sf1000}
\end{table}

\begin{table}[H]
\centering
\begin{tabular} {| l | c | c |}
\hline
\textbf{Entity} & \textbf{Num Entities} & \textbf{Bytes} \\
\hline \hline
comment & 2335637135 & 258944003306 \\
\hline
forum & 36098481 & 2222966076 \\
\hline
organisation & 7996 & 813396 \\
\hline
person & 3600000 & 324485964 \\
\hline
place & 1466 & 83793 \\
\hline
post & 555306166 & 83647390485 \\
\hline
tag & 16080 & 1122520 \\
\hline
tagclass & 71 & 4037 \\
\hline \hline
\textbf{Relation} & \textbf{Num Relations} & \textbf{Bytes} \\
\hline \hline
comment\_hasCreator\_person & 2335637135 & 69009917568 \\
\hline
comment\_hasTag\_tag & 3042978961 & 63451008509 \\
\hline
comment\_isLocatedIn\_place & 2335637135 & 44333145872 \\
\hline
comment\_replyOf\_comment & 1184778982 & 36597884006 \\
\hline
comment\_replyOf\_post & 1150858153 & 35549852967 \\
\hline
forum\_containerOf\_post & 555306166 & 16985930071 \\
\hline
forum\_hasMember\_person & 3277239057 & 167465482785 \\
\hline
forum\_hasModerator\_person & 36098481 & 1071895282 \\
\hline
forum\_hasTag\_tag & 116727525 & 2398752244 \\
\hline
organisation\_isLocatedIn\_place & 7996 & 79492 \\
\hline
person\_isLocatedIn\_place & 3600000 & 64736060 \\
\hline
person\_hasInterest\_tag & 84229044 & 1692899009 \\
\hline
person\_knows\_person & 447163916 & 22530441760 \\
\hline
person\_likes\_comment & 2858070323 & 146129764930 \\
\hline
person\_likes\_post & 1361722197 & 69623238723 \\
\hline
person\_studyAt\_organisation & 2878718 & 72544726 \\
\hline
person\_workAt\_organisation & 7829672 & 190876421 \\
\hline
place\_isPartOf\_place & 1460 & 12098 \\
\hline
post\_hasCreator\_person & 555306166 & 16467663132 \\
\hline
post\_hasTag\_tag & 793254841 & 16381717061 \\
\hline
post\_isLocatedIn\_place & 555306166 & 10543790321 \\
\hline
tag\_hasType\_tagclass & 16080 & 163488 \\
\hline
tagclass\_isSubclassOf\_tagclass & 70 & 791 \\
\hline \hline
\textbf{Property Files} & \textbf{Num Properties} & \textbf{Bytes} \\
\hline \hline
person\_email\_emailaddress & 6141306 & 276081939 \\
\hline
person\_speaks\_language & 7932926 & 144358836 \\
\hline \hline
\textbf{Total Entities} & \textbf{Total Relations} & \textbf{Total Bytes} \\
\hline \hline
 2930667395 & 20704648244 & 1066123107668 \\
\hline
\end{tabular}
\caption{ General statistics for SF 1000}
\end{table}

\begin{table}[H]
    \centering
    \begin{tabular}{|c||r|r|r|r|}
\hline    \multicolumn{5}{|c|}{SF = 1000 }  \\
\hline & Min & Max & Mean & Median   \\
\hline  \#posts/user  &1 &  1576 & 163 & 103 \\
\hline  \#friends/user  &1 &  636 & 124 & 73 \\
\hline
\end{tabular}
\caption{Detail statistics for SF 1000}
\end{table}

\begin{figure}[H]
\begin{center}
  \subfigure[{\small Cumulative number of friends.}]{\label{numFriendsCumm}\includegraphics[width=0.32\textwidth,clip]{figures/scalefactor/numFriendsCumm1000}}%
  \subfigure[{\small Histogram \#posts per user.}]{\label{numPostPerUsers}\includegraphics[width=0.32\textwidth,clip]{figures/scalefactor/numPostsUserHist1000}}%\\[1ex]
  \caption{Data distributions for SF 1000}
  \label{fig:datadistSF1000}
\end{center}
\end{figure}


\end{document}
